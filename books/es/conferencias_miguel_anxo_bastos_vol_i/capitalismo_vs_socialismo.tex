\chapter{Capitalismo vs socialismo.}

\chapterprecis{\href{https://www.youtube.com/watch?v=05VO-FLrNi8}{Comparación entre Capitalismo y Socialismo en el X Seminario InterUniversitario Economía para la Política. Universidad de San Carlos de Guatemala, Guatemala, 18 de abril de 2012.}}

\lettrine[lines=2, findent=3pt, nindent=0pt]{E}{l} tema que el profesor Cristian me propuso es el tema de los sistemas políticos económicos comparados o sea, básicamente comparar el capitalismo y el socialismo. Y después el tema de cuál es la forma la forma política pues más adecuada para promover uno o el otro. ¿Cuál es de entre las distintas formas políticas que existen hoy en día, ya se olvidó el estudio de las formas, pero las formas siguen siendo relevantes, democracia, autocracia, dictadura, monarquía, más adecuada para promover el desarrollo económico? ¿Cuál es la forma más adecuada para promover estos valores?. Entonces, es un debate a ver qué podemos sacar de aquí. Yo, básicamente no tengo una conferencia muy larga, voy a traer una serie de puntos y después los vamos a discutir.

Bien, mi tesis va a ser una y va a ser clara, primero y principal, el socialismo es malo y el capitalismo es bueno, capitalismo del libre mercado, es así de fácil si aprenden esto, ya aprendimos bastante. El socialismo es malo, copien ahí, y el capitalismo del libre mercado es bueno, es la única opción y dirán esto: \enquote{¿Y por qué?} Yo así en mi modestia apunto 4 rasgos.

Primero, el socialismo no es libre el socialismo es totalitario. \enquote{¿Por qué?}, me dirán ustedes, porque sin propiedad privada no hay libertad, sin propiedad privada no puede haber libertad. No puede haber libertad de expresión, libertad de asociación ni libertad de movimiento ni ningún otro tipo de libertad humana. Por qué no puede haberla esto lo explica una serie de autores estoy pensando en un libro célebre que es el ``Camino de Servidumbre``\cite{hayek1944road} de \href{https://en.wikipedia.org/wiki/Friedrich_Hayek}{Friedrich von Hayek}\footnote{Acádemico británico nacido austríaco ganador del premio Nobel de Economía.}, un libro clásico que apunta este tema explicando que incluso el socialismo y cuando habla de socialismo habla de todos los tipos de socialismo no solo de socialismo marxista sino de la socialdemocracia, del intervencionismo y de todas las formas, digamos que incorporan algún tipo de coerción institucional sobre los seres humanos impidiendo a los seres humanos comerciar o actuar libremente entre ellos. Hayek en ``Camino de Servidumbre``, \href{https://en.wikipedia.org/wiki/Wilhelm_R%C3%B6pke}{Wilhelm Röpke}\footnote{Economista alemán, uno de los padres del ordoliberalismo.} en ``La crisis social de nuestro tiempo``\cite{röpke1950social}, o por ejemplo, en la obra de \href{https://en.wikipedia.org/wiki/Ludwig_von_Mises}{Mises}\footnote{Economista austríaco referente de la escuela austríaca de Economía.}, por ejemplo, ``Liberalismo``\cite{vonmises1995liberalism} o por ejemplo, en la obra de \href{https://mises.org/mises-daily/who-was-gottfried-dietze}{Gottfried Dietze}\footnote{fue Profesor de Ciencias Políticas en la Universidad John Hoptkins.} como en ``En defensa de la propiedad``\cite{dietze1963defense}, Dietze es un autor alemán emigrado de Estados Unidos también que toca este tipo de tema. ¿Y por qué no puede haber libertad en el socialismo? Libres somos todos dentro del interior de nuestras mentes, dentro del interior de nuestras cabezas todos somos libres, en nuestras ideas, en nuestra propia persona no puede mandar nadie, interiormente, internamente cada uno de nosotros tenemos las ideas que queremos,no hay forma de cambio. Lo que sí se nos puede impedir es expresarlas, lo que sí se nos puede impedir es llevarlas a cabo y para eso, para poder llevarlas a cabo, para poder expresarlas sí que necesitamos de libertad, sí necesitamos de propiedad privada, es decir, la liberta, la expresión de la libertad requiere siempre de una base material. Quiero decir para poder expresar la libertad necesitamos pues, un medio de comunicación, necesitamos de papel, necesitamos de tinta, necesitamos de una imprenta, necesitamos de una logística de distribución, o sea, necesitamos de bienes materiales. Si queremos explicar en un auditorio, un debate de este estilo, necesitamos un auditorio, un local y necesitamos un sitio, electricidad, luz, agua, etc. que nos permitan llevar a cabo este bien. Si queremos hacer un programa de televisión, un programa de radio, necesitamos una emisora de radio necesitamos un receptor de radio, necesitamos un local donde hacer el programa, necesitamos unas cámaras para filmar y emitir el programa, si queremos incluso usar internet necesitamos de computadores,
necesitamos de edificios y necesitamos de lugares para hacer este tipo de cosas.

Si el Estado posee todos los medios de producción si el Estado posee todos los bienes de una nación simplemente si queremos hacer un programa de radio ¿Dónde lo hacemos? Si queremos elaborar un discurso como este ¿Dónde lo hacemos? Necesitamos pedir permiso, por ejemplo, ¿Qué hizo \href{https://en.wikipedia.org/wiki/Salvador_Allende}{Salvador Allende}\footnote{Político socalista chileno, Presidente de Chile desde 1970 a 1973.} en su momento? Pues no hizo nada raro. ¿Puso censura? No, no puso censura simplemente nacionalizó el papel de prensa y nacionalizó los repuestos de radio, no hizo falta censurar, no hizo falta elaborar una ley de censura , una ley de mordaza, nada así por estilo, simplemente nacionalizó el papel de prensa y simplemente a partir de entonces sólo le daba papel a los suyos. ¿Prohibió algo? No, pero simplemente privó de la base material a la libre expresión y en general, cualquier reunión, cualquier asociación, cualquier libertad, sea hasta de trabajo, si todos los medios de producción son del Estado y no me gusta mi trabajo a dónde voy, a dónde voy si todas las empresas son del Estado. Entonces, en primer lugar, creo que es algo obvio, si todo es propiedad del Estado, si hy una sociedad socialista y todo es propiedad colectiva, la libertad depende de la voluntad, del capricho, del arbitrio del gobernante, es decir, el gobernante quién decide qué libros se editan y cuáles no se editan, qué música se puede ejecutar o qué conciertos se pueden dar y cuáles no y qué conferencias se pueden dar y cuáles no, qué religiones pueden existir y cuáles no. Entonces, primer punto, si no hay propiedad privada no hay libertad porque les repito la libertad requiere siempre de una base material la expresión de la libertad política o económica requiere de base material. Igualmente en el ámbito de la libertad de empresa si todo es propiedad del Estado yo quiero iniciar un negocio, yo quiero comprar una cosa, vender una cosa, necesito de algún tipo de autorización administrativa que me puede ser dada o negada, este el primer punto a considerar el socialismo no puede ser libre porque desde el momento que ustedes tienen qué pedir permiso para hacer algo no son libres.

Segundo punto. El socialismo es explotador el capitalismo no. El socialismo es explotador porque tiene qué ser explotador. A ver, supongamos cuando un socialista o un marxista critica el sistema de mercado no lo critica por el fin que le da el dinero, es decir, cuando por ejemplo, \href{https://en.wikipedia.org/wiki/Bill_Gates}{Bill Gates}\footnote{Empresario y filántropo estadounidense fundador de Microsoft.} dedica mucha buena parte de su dinero a la caridad, a la investigación en medicinas o a la investigación en ayuda a los pobres, dedica buenas sumas de dinero, pero a Bill Gates no se le critica por eso, le dicen \enquote{No, es que el origen de su dinero, está en la explotación} que después lo gaste bien o lo gaste mal es indiferente, es decir, para el marxista es indiferente cómo el capitalista gasta el dinero para él la esencia del mal está en cómo él obtiene el dinero, supuestamente entre comillas pagándole algo menos de lo debido a los trabajadores. Eso es falso. El socialismo, vamos a ver una sociedad socialista como las que existieron en el mundo tenemos Cuba o Corea del Norte o la Unión Soviética antigua o los países del este de Europa en sus tiempos son sociedades socialistas. Esas sociedades ¿No tenían armamento? ¿No tenían ejércitos? ¿No tenían sistemas de previsión social? ¿No tenían sistemas de todo tipo? ¿No hacían calles? ¿No tenían bienes públicos? Y esos bienes públicos ¿Cómo se financiaban en un sistema socialista? ¿Cómo se financiaba el ejército de la Unión Soviética? ¿Cómo financiaron el Sputnik? ¿Cómo financiaron las bombas atómicas? Pues pagándole menos a los obreros si no ¿Cómo se puede permitir un excedente? Para que un país socialista tenga excedente para hacer determinados bienes, diplomacias, ejércitos, todo tipo de burocracias, todo tipo de gente, es necesario pagarle al trabajador menos de lo que realmente cobra o menos de lo que realmente vale su trabajo si no ¿Cómo podría financiarse todo este tipo de cosas? Esos países socialistas, por definición, no hay impuestos ya que todo es propiedad del Estado quiere decir que los salarios que reciben los trabajadores en el socialismo si se quieren financiar determinados bienes públicos tienen que ser que ser financiados a costa del salario de los trabajadores no queda otra que yo sepa. El socialismo le paga al trabajador menos de lo que él merece, tiene que ser así por definición cosa que en el capitalismo de libre mercado al no existir bienes públicos o no existir ningún tipo de intervención estatal el trabajador cobraría siempre, siempre en cada momento lo que él merecería, en el socialismo no, porque hay una explotación forzada de unos colectivos a otros, es decir, la \href{https://en.wikipedia.org/wiki/Dacha}{Dacha}\footnote{Casa rural normalmente segunda residencia o para todo el año.} de \href{https://en.wikipedia.org/wiki/Leonid_Brezhnev}{Brézhnev}\footnote{Político que sirvió como el quinto Secretario General del Partido Comunista de la Unión Soviética.}, la Dacha de \href{https://en.wikipedia.org/wiki/Konstantin_Chernenko}{Chernenko}\footnote{Séptimo Secretario General del Partido Comunista de la Unión Soviética.}, los palacios que tenían en el mar Negro los jerarcas soviéticos, las limusinas, los \href{https://en.wikipedia.org/wiki/ZiL}{ZIL} aquellas limusinas enormes que tenían los soviéticos ¿A costa de quién fueron sacadas? A costa de pagarle realmente al trabajador menos de lo que vale con salarios forzosos porque, como dije antes, en el socialismo los salarios no son libres, yo no puedo mudarme de lugar si no me gusta mi ciudad, si no me gusta la ciudad de Guatemala no puedo irme a Antigua allí necesito de permiso, si no me gusta trabajar en mi fábrica no puedo cambiarme de fábrica estoy forzado a trabajar en mi fábrica y por el salario que el dictador o el jerarca del partido socialista me está pagando, es decir, cualquier bien público, cualquier bien que disfruten las clases políticas es sacado a costa de la piel del trabajador, no queda otra forma. Como decíamos antes si lo que se critica es la forma en qué se obtiene, no en qué se gasta, el sistema socialista es un sistema explotador por definición. Además al fijarse los salarios, no por el libre juego de la oferta y la demanda, no por la voluntad recíproca de ambas partes, sino al ser fijado por una de las partes hay un sistema de explotación de tal forma que unos trabajadores ganan más de lo que deberían y otros menos de lo que deberían porque los salarios son fijados políticamente y los que cobran menos están financiando a los que cobran más, no queda otro remedio, el sistema socialista, por definición, salvo que no tenga ningún tipo de bien público, salvo que no tenga ningún tipo de previsión, salvo que no tenga ningún tipo de ejército, es necesariamente explotador, es más, es el sistema más explotador que existe mucha diferencia frente a los demás.

Tercero, el socialismo es injusto, es radicalmente injusto por naturaleza. Básicamente, porque no da a cada uno según lo que merece quiero decir, cobra lo mismo el que trabaja que el que no trabaja si es un sistema de igualdad salarial salvo que empecemos a introducir mecanismos de corrección capitalistas como salarios distintos, incentivos y otro tipo de formulaciones de este estilo pero el sistema socialista tiene que ser por naturaleza injusto, cada persona no recibe lo que merece porque no lo sabe sino que se asigna políticamente los salarios, las recompensas, los premios, por méritos pues de afinidad política o por méritos de otro estilo pero desde luego no responde a lo que las demás personas piensa que merece sino que responde a algún tipo de decisión política. Entonces, el socialismo es injusto porque no premia a cada persona según lo que hace sino que da a todos igual y no hay nada más injusto que tratar igualmente a los desiguales es como si yo pongo la misma nota a todos mis estudiantes independientemente de lo que hagan, independientemente de lo que trabaje, si yo pusiera un aprobado a todos al que no fue a clase, al que estuvo durmiendo, al que estuvo en clase trabajando que se esfuerza, que se levanta a su hora, que cumple con sus deberes ¿Estoy siendo injusto o no estoy siendo injusto? Pues un sistema que premia y recompensa a todos por igual es un sistema en esencia radicalmente injusto, aparte de la injusticia que radica en que los medios socialistas son medios políticos, quiero decir desde el momento en que hay que usar un aparato de coerción, desde el momento que hay que forzar a las personas a hacer algo, desde el momento en que a las personas no se le permite hacer algo, se está usando violencia ilegítima contra esas personas. Por tanto, es un sistema radicalmente injusto porque no le deja a las personas hacer lo que ellas entienden pertinente en cada momento sino que hay que forzarlas, hay que coercionarlas, hay que impedirles que se muevan,hay que impedirles... La idea propia de planificación, la idea propia de centralización, la idea propia de que hay un plan central, la idea de que hay un consejo superior del plan que decide, es impedir la libertad de las personas se trata de que hay una racionalidad por encima de la racionalidad individual a la cuál deben supeditarse los individuos. Por tanto, es forzoso, es injusto, porque impide a las personas hacer aquello que desean, estudiar lo que desean, vivir como desean, pensar como desean, es un sistema radicalmente injusto de origen. Y además, y esto si cabe es lo principal, el socialismo no puede existir, no es que sea bueno ni malo, no puede existir, es un sistema que es imposible como nos demuestra bien la historia de los sistemas económicos. ¿Por qué?, porque es un sistema que no puede calcular. No sé si ustedes están familiarizados, algunos sí, con el debate sobre la imposibilidad económica del socialismo. En los años 20, un conjunto de estudiosos entre los que destaca \href{Boris Brutzkus}{Boris Brutzkus}\footnote{Economista ruso.}, un historiador ruso, \href{https://en.wikipedia.org/wiki/Max_Weber}{Maximilian Weber}\footnote{Sociólogo alemán, figura central en el desarrollo de la Sociología.}, un historiador y politólogo y sociólogo alemán, en un libro, no sé si alguno de ustedes lo conoce que se llama ``Economía y Sociedad``\cite{weber1921economy}, un primer ministro de Holanda llamado \href{https://en.wikipedia.org/wiki/Nicolaas_Pierson}{Nicolaas Pierson}\footnote{Primer ministro países bajos.} y un economista austríaco llamado Ludwig von Mises, estos 4 autores formulan empiezan a ver, a debatir, la posibilidad de si un sistema socialista es posible o no, no si es bueno o malo, no si es justo o injusto, sino si puede existir o no un sistema socialista y la conclusión es que independientemente aunque fuera el sistema más ético del mundo, aunque fuera el sistema más explotador del mundo, aunque fuera el sistema más justo del mundo, un sistema socialista no puede existir porque sin propiedad privada no hay precios, sin precios no hay cálculo económico y sin cálculo económico un sistema económico se derrumba, no puede ser coordinado, es decir, el socialismo es imposible. Es imposible, es un sistema que al impedir la libre actuación empresarial y al impedir la formación de precios no puede funcionar. Ustedes, cada uno de ustedes,tiene en su cabeza entre 8, 10, 12 ó 15.000 palabras todas las personas tenemos en nuestras cabezas 8, 10 ó 12.000 palabras dependiendo de la cultura que tengamos usamos esas palabras para comunicarnos, esas palabras nos permiten pedir cosas, hacer acciones y incluso pensar abstractamente conceptos como \enquote{ideal}, como conceptos ideales como \enquote{justicia}, como \enquote{igualdad}, como \enquote{belleza}, cosas por el estilo podemos usar porque tenemos conceptos en la cabeza. Por tanto, funcionamos en el mundo con palabras, usamos palabras y conceptos. No le damos importancia ninguna a esas palabras porque desde muy pequeñitos cada día aprendemos 4, 5, 6, 7 palabras y todos los días nos van metiendo palabras en la cabeza y todos los días somos un poco más sabios y aprendemos más palabras y como las llevamos siempre con nosotros mismos en nuestra cabeza no le damos importancia, lo que no le dan importancia ninguna es que igual que tienen 10 o 12.000 palabras en su cabeza tienen entre 8, 10 ó 15.000 precios y no les dieron importancia ninguna tampoco porque cada uno de ustedes desde pequeñito sus papás, sus mamás, le fueron diciendo conceptos como esto: \enquote{caro}, \enquote{barato}, por ejemplo, y se lo va diciendo, es decir un kilo de guacamole cuesta esto, un kilo de azúcar cuesta lo otro, un galón de gasolina cuesta esto, una sudadera cuesta esto, y así todos los días, la sal, el azúcar, el aceite, cuánto cuesta una universidad, cuánto cuesta un parqueo y todos los días van aprendiendo precios nuevos y también tienen 10 ó 12, 15.000 precios en la cabeza sin esos precios señores estarían ciegos, no podrían calcular, estarían absolutamente ciegos, no podrían hacer nada de nada porque buena parte de sus orientaciones, buena parte de lo que a ustedes los mueve y los lleva a andar por el mundo son precios. ustedes cuando por ejemplo escogen una carrera para estudiar escogen calculando precios entre otras cosas cogen una satisfacción psíquica pero también escogen, calculan, especulan, con lo que pueden ganar en el futuro, calculan, pues bueno, si escojo esta carrera pues voy a ganar esto y esto y esto y el trabajo va a ser este y este, si escojo esta otra carrera mi expectativa de ganancia va a ser esta, esta, esta y otra y ustedes orientan su mundo por ese tipo de precios, por ese tipo de cálculo. Y \enquote{¿Cuánto me cuesta esta carrera?} pues me cuesta esto, esto, y esto \enquote{¿Cuánto me cuesta este viaje?} esto, esto y esto, me compensa hacer esto o me compensa hacer lo otro. Todo se mueve o buena parte de su vida se mueve por estas pequeñas señales luminosas que están en sus cerebros que se llaman precios. Sin esos precios no valen ustedes para nada. Ustedes, para ir de aquí, por ejemplo, a Antigua pueden ir de muchas formas, pueden coger el celular, llamar y que venga un helicóptero a buscarlos, pueden alquilar una limusina, pueden ir en carro, pueden ir en un transporte público, pueden ir andando, pueden ir en bicicleta, según la prisa que tengan o las ganas que tengan de llegar pronto a Antigua van a usar un medio de locomoción u otro dependiendo de la urgencia que tengan. Si ustedes están muy enfermos y tienen un ataque, probablemente cojan un taxi, una ambulancia, alquilen un vehículo rápido. Si quieren pasear igual pueden ir andando, hacer una caminata, pueden ir en bicicleta. Si quieren ir normalmente pues van en autobús. Si tienen mucha prisa porque tienen, por ejemplo, una donación de riñón pueden incluso alquilar un helicóptero. En muchos países, en mi país, se hace si una cosa es muy urgente la llevan en helicóptero dependiendo ¿Y si no supiéramos los precios? ¿Y si no supiéramos supiéramos nada de nada de lo que cuestan las cosas? ¿Por qué no llamar un helicóptero? Ustedes me dirán \enquote{no, porque sabemos que es muy caro} ¿Y por qué saben que es muy caro? Porque tienen precios en la cabeza, todos tienen precios en la cabeza. Ahora, imaginen un mundo sin precios o un mundo con los precios mal puestos, como hacían en Polonia. Polonia, por cuestiones políticas el pan era más barato que el trigo, dirán: \enquote{Qué bien, que social era el gobierno polaco comunista}. Bueno, ¿Qué creen que comían los cerdos en Polonia pan o trigo? Pan. ¿Y no es un despilfarro darle pan fresco a los cerdos? Quiero decir, que este tipo de desbarajustes se llegaba porque no había precios y como no hay precios no hay ningún tipo de racionalidad y este es el problema entonces. Las sociedades socialistas se derrumbaban por falta de precio.

Sólo hubo un experimento, que yo sepa, a gran escala de abolición total del mercado de abolición total de los precios fue el llamado \href{https://en.wikipedia.org/wiki/War_communism}{Comunismo de Guerra}, de 1918 a 1922 si no recuerdo mal, instaurado por un economista, por cierto, educado en la Escuela Austríaca, que se llamaba Bukharin, no sé si alguno lo conoce. \href{https://en.wikipedia.org/wiki/Nikolai_Bukharin}{Bukharin}\footnote{Revolucionario ruso y teórico marxista.} era un economista, estudió en la Escuela Austriaca, fue discípulo de \href{https://en.wikipedia.org/wiki/Eugen_von_B%C3%B6hm-Bawerk}{Böhm-Bawerk}\footnote{Economista de la escuela austriaca.} en Viena y escribió un libro que se llamaba ``El ABC del Comunismo``\cite{bukharin1920abc} que explicaba allí cómo había que instaurar la sociedad socialista y lo primero que hicieron en la Unión Soviética en el año 18 antes de la guerra curiosamente, no fue culpa de la guerra, antes de la guerra instauraron una colectivización total de los medios de producción y abolieron todo tipo de precios. Bueno, fue la caída de producción más grande datada desde que hay historia escrita, es decir, nunca hubo una caída de la producción en un país en un año más grande que en la Unión Soviética en aquellos años. \href{https://en.wikipedia.org/wiki/Alexander_Nove}{Alec Nove}\footnote{Fue Profesor de Economía de la Universidad de Glasgow y una autoridad en historia económia de Rusia y la Unión Soviética}, por ejemplo, que es un autor marxista en ``La Historia Económica de la Unión Soviética``\cite{nove1992economic} lo reconoce y es un autor marxista, es decir, da los datos de producción del año 17 y los del año 18, por ejemplo, si producía 100 de trigo se pasó a producir 1, si producían 100 de algodón antes pasaron a producirse 2 en cuestión de un año. Nunca hubo una situación de caída tan grande de la producción como aquel año. Tan grande fue el desastre que \href{https://en.wikipedia.org/wiki/Vladimir_Lenin}{Lenin}\footnote{Teórico y revolucionario ruso fundador y cabecilla de la primera Rusia soviética.} y el mismo Bukharin tuvieron que rectificar e instaurar un modelo de política económica que se llama \enquote{\href{https://en.wikipedia.org/wiki/New_Economic_Policy}{La NEP}}, \enquote{La Nueva Política Económica} que más o menos restauró una suerte de mercado y esa suerte de mercado aunque volvió a colectivizar un poco nunca se llegó al extremo actual o al extremo antiguo del socialismo. ¿Por qué? Porque un socialismo puro no puede funcionar, no puede funcionar porque no tiene precios. Y me dirán ahora, me levantarán la mano los marxistas de aquí de esta sala \enquote{¿Y la Unión Soviética no funcionó tantos años? ¿Y cómo hacían los precios en los tiempos de la Unión Soviética tantos años como funcionó?} Los soviéticos, Gorbachov y sus asesores, cuando le preguntaron cómo hacía para poner los precios decía: \enquote{fácil, fácil}, cogían el catálogo de \enquote{\href{https://en.wikipedia.org/wiki/Sears}{Sears Roebuck}\footnote{Cadena estadounidense de tiendas departamentales.}}, unos supermercados americanos y resolvían las ecuaciones. ¿Saben la propaganda que viene de los grandes almacenes a sus casas? un kilo de de chiles 2 Quetzales, una sudadera 40 Quetzales, cogían esos precios y los resolvían \enquote{mira qué fácil} decían ellos. O cogían el Wall Street Journal miraban las tablas donde venían allí los precios, precio del azúcar, los precios mundiales del azúcar, del algodón, del trigo y tal, miraban aquellas tablas y resolvían las ecuaciones. Resolvían con mercados negros, no sé si saben qué es eso, mercados negros, es decir, en la Unión Soviética los países comunistas hay mercados negros la gente compra y comercia por fuera del mercado y eso servía para poner los precios. Y tercero, con memoria, es decir, en última instancia el sistema socialista sigue siendo deudor del sistema capitalista, es decir, que no lo quiera Dios nuestro señor si llegara algún día el comunismo, ustedes se acordarían, es decir, a día de hoy saben los precios y los tienen en sus cabezas, si llegara el comunismo aún se acordarían de que el oro vale más que la plata, de que un carro Bentley vale más que un Fiat, etc., se acordarían de ese tipo de cosas, que un viaje en helicóptero vale más que uno en un carro, se acuerdan de esa cosa y eso les permite operar, cada vez peor, pero la memoria funciona y eso es lo que permitió al sistema socialista renquear y sobre todo le permitió renquear la existencia de países capitalistas. Los teóricos del socialismo, alguno de ellos, conscientes de este problema llegó a proponer que cuando triunfara el socialismo a nivel mundial dejar a Nueva Zelanda de país capitalista para que pusiera los precios porque si no, era imposible, tenían que dejar en el mundo del comunismo una isla capitalista para poner los precios para los demás si no era imposible. Entonces, el socialismo a nivel global, a nivel total no puede funcionar, es imposible si quieren después les anticipo esto.

Entonces, ¿Qué solución nos queda? Una, capitalismo, es justo, el único sistema justo porque es el único sistema que se basa en la no coerción, quiero decir, es un sistema que se basa en el acuerdo libre entre las partes. Un viejo sociólogo alemán llamado \href{https://en.wikipedia.org/wiki/Franz_Oppenheimer}{Franz Openheimer}\footnote{Sociólogo y economista político judio alemán.} escribió un libro muy muy muy bonito que se llama ``Der Staat``\cite{oppenheimer1908state}, o sea, ``El Estado``' y en el que explica muy bien esto, es decir, hay dos formas de asignar recursos y es un señor que es socialdemócrata, que no es capitalista ni nada así por el estilo, los medios políticos y los medios económicos. Los medios políticos son los medios que usan la fuerza que usan la coerción que usan la violencia, es decir, una de las partes se impone sobre la otra, una de las partes, usa la fuerza, usa las armas, usa la fuerza y le quita parte de su dinero a ustedes, la otra no, la otra es un intercambio libre entre dos partes y cómo es un intercambio libre entre dos partes, las dos partes tienen necesariamente apodícticamente que ganar. Ese es el intercambio de mercado entonces, es el único sistema justo, es el único sistema que no usa de la fuerza, que no usa de la violencia, que no usa de la coerción y en el que las dos partes ganan y cada parte puede escoger comprar o vender a quién mejor le interese y cómo le interese, entonces el sistema es justo. Y es más, remunera a cada factor, si estudiaron ustedes Economía o estudiaron ustedes lo que es el Marginalismo saben que cada factor de producción tiene que ser remunerado por lo que vale y lo que vale es lo que deciden las demás personas que vale, el valor de cada persona es su productividad marginal descontada pero valorada por lo que valoran las demás personas, cada persona en el capitalismo cobra o tiende a cobrar lo que las demás personas entienden que merece, es duro decirlo pero es así, la gente valemos todo lo que los demás dicen que valemos, no lo que pensamos nosotros que valemos. En todos los ámbitos, es duro de asumir, hay que asumir, hay que agachar la cabeza pero qué le vamos a hacer, es así el mundo, entonces no es justo ni injusto, es así.

El sistema capitalista, como dije yo antes, no es explotador, por duro que parezca le paga a cada factor de producción lo que los demás dicen que vale. Entonces no hay ningún tipo de explotación, si quieren después entramos en el debate, pero el propio sistema marxista ahí entra en contradicción, si todos los sectores ganan lo mismo, todos lo mismo, pero cada sector económico tiene una composición orgánica al capital distinta, es decir, una relación entre capital y trabajo, y de ahí se deriva la tasa de explotación. Bien, cada sector el sector eléctrico, el sector del seguro, el sector del textil, el sector del calzado, el sector del azúcar, cada uno tiene una composición orgánica al capital, en un sector hay mucho capital, por ejemplo, el sector eléctrico tiene mucho capital, necesitan de centrales nucleares, de centrales eléctricas, de todo tipo de infraestructuras y relativamente poco trabajo. Y otros sectores, el sector del calzado, una \href{https://en.wikipedia.org/wiki/Maquiladora}{maquila}, por ejemplo, tiene mucho trabajo y poco capital no necesita de máquinas muy sofisticadas. O sea, consecuencia, composición orgánica al capital distinta, por lo tanto tasa de explotación distinta en cada sector. Entonces, la pregunta es, y los deberes que van a llevar para casa son estos, a ver si lo resuelven, porque los marxistas dieron un premio, \href{https://en.wikipedia.org/wiki/Eduard_Bernstein}{Bernstein}\footnote{Padre del revisionismo Marxista.} quería dar un premio a quien resolviera esta pregunta y no fueron capaces de resolverla. ¿Cómo con tasas de explotación distintas el beneficio es el mismo? Porque el beneficio, digo yo y dicen los teóricos de mi escuela, obviamente el beneficio por ahí se deduce que no depende de la explotación

Por último, el capitalismo es el único sistema por tanto que puede calcular, puede calcular porque tiene un sistema de establecer precios por tanto puede establecer y sobre todo tiene un sistema de beneficios y los precios y los beneficios son como luces que nos orientan, son como faros en la noche, el beneficio alto nos dice \enquote{inviertan aquí su capital tiene que quitarse dónde gana menos hacia dónde gana más} y es un proceso de perpetua evolución el capital está siempre invertido de la mejor forma posible si no hay restricción, siempre está invertido de la mejor forma posible. Y los precios siempre reflejan toda esa enorme cantidad de información que hay detrás. Los precios nos dicen qué hay que producir, cuánto hay que producir, cómo hay que producirlo, dónde hay que producirlo, de qué calidad, qué cantidad, quién debe producirlo, obreros muy cualificados y de poco salario u obreros muy cualificados y de alto salario, eso todo nos lo dicen los precios cuando se permite el cálculo y se permite un sistema racional de producción, el único sistema racional de producción. La alternativa es quedar en manos de un dictador que decida así los recursos de unos para otros. Estos son los dos sistemas.

Ahora, ¿Qué sistemas políticos favorecen esto o lo otro? El capitalismo, en general, es indiferente también a la forma, en el sentido de que puede prosperar pues en autocracias, puede prosperar en monarquías, de hecho, probablemente las formas mejores para que el capitalismo prospere son las monarquías. Las monarquías del Antiguo Régimen o sistemas de este estilo, puede progresar en democracias y puede progresar incluso en regímenes comunistas. Los países que mejor funcionan, que mejor desarrollan al capitalismo a día de hoy son países antiguos comunistas, en ese aspecto sí que son positivos, es decir, China está haciéndolo capitalistamente muy bien, Vietnam está haciéndolo muy bien,
en el este de Europa no digamos. Los países más libertarios de Europa, más librecambistas de Europa, son los países antiguo comunistas, ninguno de ellos quiso volver al comunismo, por algo será supongo. Y todos ellos son los que están adoptando las reformas más ambiciosas de Europa. Entonces hay indiferencia en la forma, lo que importa no es tanto la forma política cómo el compromiso del gobierno con los derechos de propiedad y el compromiso del gobierno con la libertad económica y entender un poco los principios económicos. En principio cualquier forma de gobierno es indiferente, ahora históricamente, se dio más en unos sitios que en otros. Yahora les voy a dejar que me interroguen si quieren.

\subsection{Pregunta: \enquote{¿De qué manera bajo un sistema capitalista se puede ayudar a la extrema pobreza a salir de ahí?}}

Estamos acostumbrados a entender la pobreza al revés de que el mundo sea rico y la pobreza sea una casualidad. La población humana vivió miles de años en situación de extrema pobreza y era pobre Inglaterra, era pobre Suiza, era pobre Francia. En Europa hasta el siglo XVIII o principios del XIX había una hambruna cada 20 años, con un agravante, por ejemplo, cuando había hambre en Galicia, en mi tierra, y había comida, por ejemplo, en Alemania, no había forma física de traer la comida de Alemania a Galicia. Estoy hablando del siglo XVIII y para atrás. En el siglo XVIII y XIX en algunos países sobre todo Inglaterra, Francia, Estados Unidos, se empieza a ver capitalismo, quiero decir, producción con capital. Lo que diferencia un país rico de un país pobre es básicamente, la dotación de capital per cápita. ¿Qué diferencia a un campesino suizo de un campesino, por ejemplo, \href{https://en.wikipedia.org/wiki/Banyamulenge}{Banyamulenge}, de las montañas Mulenge del Congo, que son de las zonas más pobres de la tierra? ¿Qué es lo que lo diferencia? ¿Por qué en Suiza no hay pobreza y en el Congo sí? Porque en Suiza, el trabajador suizo trabaja con un tractor \href{https://en.wikipedia.org/wiki/Massey_Ferguson}{Massey Ferguson} de 500 caballos, es decir, su fuerza de trabajo, sus músculos se multiplican por 1.000 o por 10.000 y en un día hace el trabajo de 200 hombres de antes, por tanto, produce trigo por como 200 veces lo que se producía antes el trabajador de las montañas Mulengue sigue trabajando como se trabajó siempre, o sea, de una forma no capitalista no capitalizada, trabaja con un buey flaco, con las manos o con un arado de madera, por tanto, su productividad es 100 veces menor. Entonces digo, en unos países de Europa se empezó a trabajar con capitales, eran países enormemente ahorradores, enormemente frugales, enormemente capitalistas, capitalizaban, ahorraban y compraban máquinas, empezaron a trabajar en vez de con las manos empezaron a trabajar con máquinas de vapor, en vez de transportar las mercancías con bueyes empezaron a transportarlas en ferrocarriles. Eso multiplicó su productividad y fue que poco a poco cada trabajador fuera siendo más rico y eso se fue expandiendo como una mancha de aceite, o sea, lo que hay que explicar es la riqueza no la pobreza, la pobreza fue siempre la condición de vida de la humanidad, la esperanza de vida de 30 años no se superó hasta el siglo XX. Sólo con medios capitalistas, de producción, de ahorro, mercado y cálculo se pudo poco a poco superar esa situación de pobreza y poco a poco la riqueza, la prosperidad, se está expandiendo por el mundo, lentamente, pero se está expandiendo por el mundo. La pobreza hoy en día es menor que nunca en la historia de la humanidad y si quieren me dicen cuándo vivía la gente mejor que ahora, incluso en los países pobres, fíjese lo que estoy diciendo. En cualquier aspecto les digo la única solución es esta. Ahora, el socialismo, miren, pongo un ejemplo, Cuba año 1958 pueden contrastarlo si tienen internet, año 58 estadísticas de la ONU, ¿Saben qué renta tenía Cuba? la de Francia y la de Austria, tenía el doble de renta que Italia y el doble de renta que España, en el año 58. Hoy en día mire, pues sí, está compitiendo con Honduras por los últimos puestos a ver cuál es el país más pobre de América Latina, ese es el logro del socialismo. Cojan el cambio al revés, en el año 60 Sudán y Corea del Sur tenían exactamente la misma renta, pueden contrastarlo también, Sudán siguió como estaba, Corea del Sur abrazó al capitalismo más o menos salvaje, multiplicó la renta por 30 y el otro quedó como estaba. Ahora Corea del Sur tiene la renta de Italia o de España y Sudán se quedó donde estaba. Esa es la solución a la pobreza, ya sé que es muy simple lo que estoy explicando necesita más tiempo, esta es la solución a la pobreza y no hay otra solución a la pobreza: \textbf{¡Capitalismo, ahorro y trabajo duro! ¡no hay otra cosa!}

\subsection{Pregunta: \enquote{¿El Socialismo tal vez no funcionó porque lo que ha habido han sido sólo experimentos socialistas?¿Hubo explotación de Europa a América latina lo que permitió a Europa desarrollar el Capitalismo?}}

O se compara el socialismo teórico con el capitalismo teórico, que fue lo que intenté hacer yo, es decir, no entré en casos concretos después puse algún ejemplo. Pero el socialismo teórico se debate contra el capitalismo teórico y el socialismo real con el capitalismo real. Lo que se hace muchas veces es criticar el capitalismo real lo que existe que es un sistema muchas veces intervenido, muchas veces con influencias socialistas, con influencia de los Estados, con una especie de socialismo ideal y eso es lo que no me gusta admitir muchas veces. Se nos critica al capitalismo real desde parámetros utópicos e ideales, no el socialismo ideal con el capitalismo ideal. Con el capitalismo ideal y el socialismo real de los países socialistas con los capitalistas reales entonces en qué... ¿en qué queremos hacer?
muchas veces se hace mucho... mucha demagogia con estas cosas el socialismo lo que critiqué yo es socialismo teórico
y digo socialismo teórico no puede funcionar independientemente que fueran ángeles los que gobernaran el...
el sistema socialista, ¿no? y no puede funcionar por lo que estuve diciendo por el cálculo porque un sistema socialista
que tenga cualquier tipo de ejército que tenga cualquier tipo de aparato estatal necesita necesariamente explotar al trabajador es el primero
entonces comparemos uno con uno y cuando me resuelvan el problema del cálculo cuando me resuelvan este tipo de cosas podemos empezar a hablar
y eso vale para cualquier forma socialista angélica, tiránica o lo que sea...
dos... la idea de que los países capitalistas se hicieron ricos por las colonias no es correcta
el saqueo se dio durante toda la historia de la humanidad los europeos fueron... fueron acosad0s por Musulmanes
fueron acosad0s por Hunos fueron acosad0s por Mongoles fueron acosad0s por Vikingos
los latinoamericanos no solo fueron acosad0s por españoles, aztecas todo tipo de gente ac0saba ac0saba a los pueblos de alrededor
los seres humanos somos todos iguales toda la humanidad es una y somos todos iguales y todos somos iguales de malos no somos unos más malos que los otros
pero aparte de eso es una falacia suponer que los países ricos se hicieron ricos a costa de las colonias
porque no es cierto no es cierto ni puede ser cierto como... por ejemplo... como Mises en "Nación, Estado y Economía"
"Nación, Estado y Economía" (1919) - Ludwig von Mises
y otros autores apuntan no puede ser correcto no puede ser correcto primer punto...
yo veo... Suiza Noruega
Suecia Luxemburgo Canadá
Australia Canadá, Australia incluso fueron... ¿qué fueron? ¿colonias o países colonizadores?
¿qué? estos países que acabo de citar ¿qué colonias tenían?
¿y no están en otros países más ricos de la tierra? el colonialismo...
y lo explicó muy bien la Escuela de Manchester es una piedra de molino al pescuezo el colonialismo es un lastre como se ve por ejemplo en el caso portugués
Portugal tuvo las colonias más grandes proporcionalmente de su época y es el país más pobre de Europa con mucha diferencia
segundo punto... dígame un solo país europeo que se derrumbara al perder las colonias
dígame un solo país europeo o colonizador que se derrumbara al perder las colonias mire...
ya le pongo ejemplo nada más el caso de España porque las colonias son un lastre de hecho, por ejemplo, Inglaterra
tenía colonias pero no invertía en sus colonias tiene los datos ahí... creo que es Lionel Robbins
"A History of Economic Thought" (1998) - Lionel Robbins
que lo historia... no... Gran Bretaña prefería invertir en Argentina o en el Brasil
o en Rusia antes que invertir en sus propias colonias las colonias es una cosa de prestigio político es una cosa de imperios es una cosa del Estado
es una cosa socialista porque socialismo también se da entre país socialista
en el país socialista, eh... que hubo guerras también imperialistas, es decir pero aparte de eso el colonialismo no da riqueza ninguna
el oro y la plata que se quitó de América saben ¿para qué valió? para subir los precios en Europa para nada más...
para nada más... para crear inflaciones de caballo en Europa y para financiar guerras para nada más no desarrolló ningún país europeo
de hecho, se desarrolló mucho más países europeos sin colonias que España con sus colonias todas que tenía
no le valió de nada porque la colonia normalmente económicamente es un lastre y tercero...
Réplica del Profesor Bastos (¿Cuánto vale un trabajador?)
habla usted de los salarios mínimos cuánto vale un trabajador mire, el trabajo nos guste o no nos guste
es una mercancía ustedes mismos lo valoran como mercancía ustedes cuando escogen entre un taxi
o un autobús están valorando el trabajo de ese trabajador está diciendo si valoran o no el trabajo de manejar un carro o no
entonces, un trabajador, ¿cuánto vale? vale lo que decide los demás que vale si eso cubre un mínimo o no
pues no lo sé usted hablaba de que el trabajo es duro yo no sé... como era aquí antes yo sé en Galicia mi abuelo
que en paz descanse trabajó de los 9 años desde los 9 años trabajaba en la conserva
en el campo con los animales y cultivando la tierra desde los 9 años
pero él no andaba a protestar y no tenía ni... no había domingos, eh... no había domingos mi abuelo nunca conoció un domingo
hasta muy mayor conoció un domingo porque los animales los animales comen todos los días
no sé si según ustedes tienen tienen antepasados en el campo pero los animales comen todos los días y hay que cuidarlos todos los días
y hay que ordeñar las vacas o lo que sea todos los días no había descanso ahora... gracias a que él ahorró...
gracias a que hizo cálculo económico ya... ya mi padre y ya yo ya pudimos trabajar de otra forma
pero gracias a él yo se lo agradezco pero aún así mi abuelo aún estando mal cada día de su vida estaba mejor
porque veía que el país iba prosperando y que iba avanzando y que el sistema económico era el único que había
no quiso... si mi abuelo protestara que el sistema estaba mal no arreglaría nada mi abuelo lo que hizo es agachar la cabeza
trabajar duro ahorrar y ese dinero bien invertido con cálculo derivó en la prosperidad más o menos
que tenemos hoy en día no hay otro camino ¿qué pasa? que no es... es lento es lento es un proceso que lleva 50 o 60 años
no es mañana El licenciado nos hizo saber los efectos positivos del capitalismo
Pregunta 3
pero, eh, quisiera que fuera más allá en el sentido de que
si solo miramos los efectos positivos de ese sistema es como un espejismo porque solo miramos
lo que nos conviene o nos interesa entonces, quisiera que... eh, nos pudiera decir
el otro lado de la moneda porque ningún sistema es perfecto entonces, ¿cuáles son los efectos negativos
que tiene el sistema capitalista? No tiene ni puede tenerlos... mire qué fácil le respondo...
Réplica del Profesor Bastos (El capitalismo no tiene defectos)
no... si quiere después me... preferiría que me dijera usted alguno que le pareciera mal y después se discutía yo no veo ninguno...
no veo ninguno por un motivo porque el capitalismo es un sistema que depende del acuerdo libre entre las partes y si las partes se ponen de acuerdo
y lo demandan será que eso es lo que desea la gente no puede haber efecto negativo ninguno ahí
Pareciera como que la receta está lista pero la pregunta es ¿por qué en Guatemala... eh, tocando un tema contextual, digamos...
Pregunta 4
no sale adelante si vivimos en un sistema capitalista? Les aconsejo que miren un índice que es el índice "Heritage"
Réplica del Profesor Bastos (Índice de Libertad Económica)
de libertad económica a ver en qué puesto está Guatemala miren, hay un índice... lo buscan en internet
que se llama "Índice de Libertad Económica" que mide todos los aspectos de libertad económica no solo el capitalismo
por ejemplo, las regulaciones o sea, cuántos trámites necesitan ustedes antes de abrir un negocio, por ejemplo
cuánto... qué... qué seguridad tiene la propiedad o sea, ¿su propiedad puede ser expropiada a voluntad o no?, etc.
todo un conjunto de normas impuestos, regulaciones, trabas, etc. y vea que Guatemala
está entre los países menos libres del mundo y mire... y mire... pero no solo por capitalismo o no
el capitalismo aquí es un capitalismo fuertemente intervenido y cuanto más intervenido está menos se le deja
o sea, pero compare, contraste los datos A eso voy, o sea, nuestro desarrollo
Pregunta 5
académico, intelectual, científico debe ir más allá de recetas (Ininteligible)
realmente las líneas que a nosotros como sociedad, como país nos lleven a ese desarrollo
porque hemos venido con esta digamos, "máscara capitalista" que hemos tenido durante muchos años en Guatemala
y que realmente no nos han llevado nada más que a la explotación porque cómo poder construir desarrollo realmente si
el 85\% del territorio nacional está concentrado en un 5\% en la sociedad o sea, son datos que no nos permiten construirnos
y son problemas que no podemos seguir aguantando durante 60 años para que ese desarrollo se dé...
Réplica del Profesor Bastos (El costo de la vida en un país rico y un país pobre)
Mire, un detalle... un detalle... yo tengo amigos aquí españoles, obviamente
¿cómo puede ser que un carro... cueste casi un 50\% más en un país supuestamente más pobre
que el nuestro que en España? un carro... ¿cómo puede costar el mismo carro
60\% más aquí en otros países? ¿cómo puede costar cualquier bien de consumo... importado un 60\% más que en mi país?
por unos aranceles y un intervencionismo brutal eso no es capitalismo eso los empobrece a ustedes todos por ejemplo, les dificulta tener carros
por ejemplo... ese tipo de cosas no puede ser que un mismo automóvil cueste un 60\% más aquí...
y es el mismo carro porque hay alguien que le mete... que le mete aranceles en el medio diga usted...
Intervención
Acá entre nosotros nunca se hace claro, por ejemplo, ahorita qué es un capitalismo del libre mercado
y qué es un capitalismo en donde hay... empresas o sectores protegidos que es lo que pasa aquí en Guatemala
todos se cuestionan acerca de la pobreza y de la extrema pobreza en Guatemala y lo primero que hacen es decir
"es culpa de los ricos" pero hay que dividir y separar porque hay millonarios
que se han hecho millonarios por una idea que tuvieron y que pensaron, siendo pobres
viviendo de extrema pobreza hay empresas... ustedes pueden averiguar el historial de esas empresas
que empezaron de la nada y por una idea brillante hoy... satisfacen las necesidades
de muchos de nosotros aquí pagan impuestos les pagan bien a sus empleados
y es una empresa que funciona bajo el libre mercado tenemos que revisar las otras empresas
que son las protegidas por el Estado las cuales sí le han hecho daño al país pero si nos cuestionamos
yo he trabajado con gente que vive en un extrema pobreza y llego al punto a veces en el sentimiento que yo veo
cómo hay gente que tiene tanto y estas personas viven así pero si empezamos a pensar en la solución
¿cuál será la solución? ¿quitarle la libertad a todos y que todos vivan igual?
o ¿que busquemos una sociedad de libre de mercado donde todos tengamos la oportunidad de producir
de satisfacer nuestras necesidades y que no sean empresas protegidas? pero es lo que... no se nos enseña
o sea, solo el... o capitalismo o socialismo o ricos o pobres pero hay que saber dividir esas dos...
Perfecto, no tengo nada que añadir.