\chapter{La revolución francesa.}

\chapterprecis{\href{https://www.youtube.com/watch?v=-rXk8cJCdr8}{Disertación del profesor Miguel Anxo Bastos Boubeta sobre la Revolución Francesa.}}

\lettrine[lines=2, findent=3pt, nindent=0pt]{B}{uenas} tardes a todos. No me gusta que hablen castellano por mi culpa o sea por si us plau parle en Catalá a las preguntas yo no parlo bem pero gosto muito de escoltar el Catalá. Además me prometieron cuando viniera aquí iban a ser independientes que iba a ser el primer español o gallego que viniera aquí a hablar y no, aún veo, que aún no trabajaron bastante, pronto llegará.

Bien, yo voy a hablar un rato sin medida, quiero decir, uno de los problemas de la Revolución Francesa es la racionalización del tiempo y la racionalización de todo todo quiso hacerse en base decimal, yo digo hablaré el tiempo que vea yo que la conferencia me da y después ustedes me preguntarán hasta que tenga que que marchar, pero no vengo a dar una conferencia cerrada, me gusta que intervengan y pregunten y que después al final pues que podamos debatir sobre lo que aquí se dio. Yo, como acaba de explicar Martín, yo no soy historiador, entonces no quiero decir que sé de lo que no sé, soy un politólogo malo y un economista peor aún, entonces, quiero decir, de lo poco que sé es de otras cosas pero siempre me interesó la Revolución Francesa y creo que la Revolución Francesa no está bien tratada desde el punto de vista libertario, en general, como buena parte de la historia, o sea, la Historia normalmente la escribe gente que sabe mucha Historia pero tiene pocos conocimientos, por ejemplo, de teoría austriaca o de teoría política libertaria, por tanto, muchos de los análisis que hacen a mi entender pues no reflejan toda la realidad de lo que se está discutiendo Entonces yo quisiera dar aquí mi visión una visión digamos, libertaria, aunque no sea muy completa, aunque no sea, digamos, la de un historiador profesional, que no quiero meterme donde no hay, seguramente se me escapan muchos detalles aquí hay muchos debates, si fue una revolución burguesa, si fue una revolución Nacional, los debates en los que no quiero yo entrar. Muchos temas están ya están debatidos y hay mucha erudición hay interpretaciones marxistas o interpretaciones liberales, no quiero entrar yo en esos temas. Simplemente de las consecuencias que tuvo la Revolución Francesa me interesa destacar algunas, discutirlas en clave libertaria y ver qué aportaciones se pueden hacer. Repito, también es un llamamiento si alguno de ustedes estudia Historia, está interesado en Historia, que por favor escriban libros sobre Historia, hace falta mucha historia y desde otro punto de vista. Después veremos que cuando es, por ejemplo, cuando muchos historiadores analizan la Revolución Francesa y analizan las consecuencias económicas, por ejemplo, de los decretos de máximos de precios o todo toda la inflación de los asignados no aciertan muy bien a explicar lo que pasó porque creo que les falta una teoría económica que explique lo que pasaba aquellas cosas. Ellos solo explican que subían los precios, que había crisis de subsistencias, que faltaban bienes, pero no explicaban las razones de eso. Yo digo, es fácil, hombre si usted pone un precio máximo va a crear desabastecimiento ya sin problema ninguno, pero eso lo veremos ahora. Por favor, entonces si alguien está interesado en esta historia o en cualquier otro tipo de historia, por favor, que escriba hace mucha falta análisis históricos desde nuestro punto de vista de los eventos del pasado, creo que es un punto que hace falta, entonces me gustaría si hay un historiador convencerlo de que se apunte a estas a estas cosas.

La pregunta de la conferencia es si hizo la Revolución Francesa la raíz de todo mal, a ver, de todo mal, no y yo soy muy crítico con la Revolución Francesa pero todo mal no, hizo cosas buenas, por ejemplo, abolió la esclavitud, La Revolución Francesa abolió la esclavitud y los privilegios, eso hay que reconocérselo, fue la que abolió la esclavitud, lo hizo a mi entender mal porque generó una guerra terrible en Haití, generó la destrucción casi todo un país, se podía hacer, digamos, de una forma más ordenada, pero acabó la esclavitud está claro y es un mérito que se le puede achacar. Que abolió los privilegios también es una cosa que se le puede achacar o apuntar a su favor, creo que los abolió de forma al revés de cómo debería haberlos abolido, porque uno de los privilegios que tenían era que la nobleza no pagaba impuestos, tenía que ser en vez de hacer pagar impuestos a la nobleza tenía que no hacer pagar impuestos al resto y la abolición del privilegio tenía que ser al revés no haciendo pagar a todos sino extendiendo el privilegio a toda la población. Entonces, ¿Qué puntos quiero yo tocar? Repito, es una revolución muy alabada, universalmente alabada, de hecho, yo creo que si me trajeron aquí es porque alguna vez tengo discutido con colegas míos o así y yo me enfado mucho cuando hablan de Revolución Francesa, sobre todo cuando un Nacionalista o un Independentista defiende la Revolución Francesa yo me pongo de los nervios, porque es lo contrario a la Revolución Francesa. Un Independentista o Nacionalista tiene que estar con la \href{https://en.wikipedia.org/wiki/Chouannerie}{\enquote{Chuanería}}\footnote{Levantamiento antigubernamental que afectó a las zonas rurales de algunas regiones del oeste de Francia.} tiene que estar con la \href{https://en.wikipedia.org/wiki/War_in_the_Vend%C3%A9e}{\enquote{Vendée}}\footnote{Levantamiento contrarevolucionario que tuvo lugar en la región Francesa de la Vendée} tiene que estar con los contrarrevolucionarios porque eran los que luchaban contra la centralización, eran los que luchaban defendiendo las viejas lenguas los viejos parlamentos y los viejos derechos frente al Españolismo o Francesismo de París que quería uniformar todo aquello.

Entonces, repito, no todo es malo. pero ¿Por qué yo soy crítico con la Revolución Francesa? ¿Qué puntos veo yo? Primero, que instaura una República, y dirán, \enquote{no sé qué, no sé cuánto}, una República muy virtuosa. Los jacobinos y en general los revolucionarios franceses eran muy deudores, ahora tenían un pensamiento que le llaman ahora la gente moderna "Republicano", un pensamiento así del bien común, de la virtud, de una ciudadanía ilustrada y participativa valoraban mucho eso de "la incorruptibilidad", de ahí tenemos a \href{https://en.wikipedia.org/wiki/Maximilien_Robespierre}{Robespierre}\footnote{Líder prominente de la Revolución Francesa.} que le llamaban "El Incorruptible", pero abolió la Monarquía en nombre de una especie de República. Y eso, ¿Fue bueno o fue malo? Yo soy Monárquico como supongo que los catalanes, los catalanes siempre estuvieron en un Principado nunca fue una República ni una cosa así, que sepa es el Principado de Cataluña, igual que Galicia es el reino de Galicia. Entonces, ¿Por qué defiendo la Monarquía? La Monarquía a mí me parece siempre una forma de gobierno bastante civilizada y en los tiempos de Luis XVI era bastante civilizada,  sí, había \href{https://en.wikipedia.org/wiki/Lettres_de_cachet}{\enquote{letras de cachet}}\footnote{Cartas que servían para transmitir una orden del rey.}, sí, había opresión, pero estaba muy atenuada, estaba entrando en un proceso de civilización que estaba Francia progresando, las libertades estaban creciendo y el país estaba prosperando lentamente. Yo siempre pongo un ejemplo, vamos a ver si los ilustrados, \href{https://en.wikipedia.org/wiki/Voltaire}{Voltaire}\footnote{Defensor de la libertad de expresión, libertad de religión y separación de la Iglesia y el Estado.} y \href{https://en.wikipedia.org/wiki/Jean-Jacques_Rousseau}{Rousseau}\footnote{Escritor, filósofo y compositor.} y \href{https://en.wikipedia.org/wiki/Baron_d%27Holbach}{D'Holbach}\footnote{Filósofo, enciclopedista y escritor, prominente figura de la Ilustración francesa.}, \href{https://en.wikipedia.org/wiki/Montesquieu}{Montesquieu}\footnote{Jurista, intelectural y filósofo político francés deefensor de la separación de poderes.} es anterior, pero toda aquella gente de las enciclopedias, \href{https://en.wikipedia.org/wiki/Denis_Diderot}{Diderot}\footnote{fue una figura decisiva de la Ilustración como escritor, filósofo y enciclopedista francés.}, pero vamos a ver, esta gente ¿Dónde escribía? ¿Dónde estaba? ¿Esos libros estaban quemados? ¿Fueron perseguidos? ¿O esa gente eran los círculos intelectuales dominantes en el París de esa Monarquía tan atrasada que había? Es decir, esos intelectuales tan ilustrados, tan tal, ¿Dónde escribían? ¿Escribían en las mazmorras o eran los dominantes intelectualmente en su tiempo? O sea, ¿Dónde estaba esa supuesta represión que había en aquella época? Que es que la había obviamente, no era un régimen ideal y te podían meter preso, pero ninguno de estos así líderes pues lo pasó especialmente mal, unos peor que otro, pero no fue una persecución, no eran unos libros que estuvieran quemados o que te condenaran a la hoguera por leerlos, es decir, era un mundo intelectual que era bastante próspero, como en general, son las monarquías.

Las monarquías es una forma de gobierno, después podemos discutirlo, bastante civilizado y ¿Por qué critico la formación de la República? Porque la monarquía es una forma ordenada. Cuando llega la República lo que incitaron inmediatamente es una forma de gobierno muy inestable. Fíjese por ejemplo, en España, durante todo el tiempo de la monarquía no había golpes de Estado, era un gobierno bastante ordenado y que llevaban lentamente un proceso de civilización, incluso por la vía de la Ilustración en cuanto se acaba con esa forma de gobierno todo gobierno se convierte inmediatamente en inestable, en ilegítimo, en un gobierno que de base o raíz o legitimación, aún no siendo un gobierno popular se transforma en un régimen de base popular. Entonces, ¿Qué empiezan? Los pronunciamientos, los golpes de Estado, todo tipo de motines, todo tipo de cosas, genera gobiernos muy inestables y yo creo que ese tipo de formas de gobiernos inestables derivan en inestabilidad política, derivan en desorden y derivan en un proceso de "descivilización".

Cuando yo hablo del proceso de civilización, me refiero a los conceptos que usa, por ejemplo, \href{https://en.wikipedia.org/wiki/Norbert_Elias}{Norbert Elias}\footnote{Fue un sociólogo alemán de cultura judía, cuyo trabajo se centró en la relación entre poder, comportamiento, emoción y conocimiento.} en su libro ``El Proceso de Civilización``\cite{elias1939civilizing} es un libro que explica que las sociedades lentamente se van haciendo más civilizadas, más pacíficas, más prósperas, a través del intercambio, a través del comercio, a través de las interacciones y a través, incluso, de las normas de urbanidad costumbres y buenas maneras y es un proceso lento. La monarquía contribuye a eso en el sentido de que es un poder relativamente blando con respecto a los demás poderes, no defiendo la monarquía, defiendo la monarquía en relación a los demás poderes como ya explicaré, yo soy un anarquista. Entonces, no soy monárquico, pero sí que defiendo que la monarquía es más próxima a la anarquía que puede ser otras formas de gobierno por paradójico les parezca a ustedes, después lo podemos discutir, soy un anarquista, digamos, un poco peculiar. Pero la idea es esa, se plantea una idea de acabar con toda forma de poder, se plantea una situación de inestabilidad, miren cuántos gobiernos hubo en la Revolución Francesa, en 3 ó 4 años hubo dos Constituciones, una de ellas no aprobada
a pesar de ser ratificada por la voluntad expresa de Robespierre, hubo Constituciones, hubo un montón de gobierno, los gobiernos los decidían las masas en las calles echando o subiendo o bajando gobiernos por manifestaciones en las calles, es decir, no fue una sociedad especialmente ordenada. Entonces digo, la monarquía, que era un régimen que pensaba a muy largo plazo porque la preferencia temporal de las monarquías es muy baja, ese proceso se alteró de forma brutal y pasó a acelerar o a cambiar el tiempo. ¿Qué hubiera pasado de no haber una revolución? Que las monarquías probablemente evolucionarían lentamente y se haría un régimen de libertades, los régimenes seguramente evolucionarían de una forma parecida a lo que es a día de hoy el gran ducado de \href{https://en.wikipedia.org/wiki/Liechtenstein}{Liechtenstein}, por ejemplo, o al principado de Mónaco, serían formas políticas de ese estilo de corte monárquico, pero con grandes libertades, sin opresión y con Estados muy pequeños y poco interventores y con un grado enorme de libertad civil y de participación, etc. Eso es lo es es lo es el proceso que creo yo que se trunca con la Revolución francesa, se pasa de un gobierno civilizado que piensa a largo plazo con muy poca intervención, porque la monarquía tiene una ventaja muy grande que es muy poco legítima y esa ilegitimidad, es lo que me gusta a mí, la gente ve el gobierno monárquico como algo ilegítimo porque es ilegítimo, pero lo ve de forma clara, entonces se oponen de forma tajante al poder. La República, en cambio, es una forma de gobierno altamente ilegítimo por lo tanto la gente acepta ese poder y acepta el poder estatal sin tanta precaución como con los demás poderes. Es una paradoja, pero es así, por eso me gusta, no es que me gusta la monarquía es que la gente se revuelve más frente a la monarquía que frente a otras formas de gobierno y las frena, por eso es raro que un Rey cobre más de un 10\% de impuestos, como fue así históricamente, los Reyes nunca cobraron impuestos porque la gente se levantaba y les quemaba las casas y los Palacios. Las repúblicas modernas, en cambio, pueden cobrar 50\% de impuestos y la gente pide más.

Además instauró formas de aristocidio. A ver, todo estado moderno busca acabar con sus enemigos para concentrar el poder, el primer poder que elimina las repúblicas son las aristocracias y en en Francia, no sé si se puede denominar o no, pero hubo algo parecido a un aristocidio, un asesinato más o menos con la guillotina de buena parte de la élite gobernante existente. Esto genera inestabilidad y genera también una concentración del poder en las nuevas clases gobernantes que se discuten si vienen o no de la burguesía. Pero se trata de eliminar el primer obstáculo, el primer obstáculo del poder central es la existencia un grupo de personas dotadas de cohesión, dotadas de valores propios y con capacidad digamos de de liderazgo social. Entonces lo primero que hace la mayoría de las revoluciones son aristocidios, acabar con los elementos dominantes del Antiguo Régimen
para poder instaurar sin control el nuevo poder.

Dos, ¿Cuál es el otro enemigo del poder central? Los poderes locales. ¿Qué hace la Revolución francesa, lo primero que hace? Acabar con los parlamentos, acaba con los poderes locales, centraliza los poderes locales, elimina, por ejemplo, los parlamentos todos que había en Francia los quita, implanta el idioma único francés, cuando fue la Revolución francesa, el tipo de revolución francesa, solo 20\% ó 25\% de los franceses hablaba francés, pero se instaura por la fuerza el idioma francés supuestamente para eliminar desigualdades, para que todos los franceses se entiendan entre ellos, entoncesse acaba con las 6 lenguas que hay en Francia, el vasco, el catalán, el Bretón, el occitano y el alemán de Alsacia aparte del francés, pues las liquida, primero las transforma en patuás, las transforma en idiomas así acastrapados, idiomas así degradados hasta que después las elimina a través de la educación pública que para eso se se inventa también. Pero básicamente centraliza el poder porque la gente, las personas, en aquellos tiempos normalmente eran más leales a su territorio, a su a su nación de origen, su viejo territorio, que a Francia. Se trataba de que los franceses fueran leales en primera instancia al gobierno central de París y para eso hay que eliminar todos los poderes locales. Entonces se centralizan, se divide Francia en departamentos, los departamentos son unidades administrativas nuevas creadas de forma geométrica, o sea, la idea de geometría es una idea ilustrada, en España, \href{https://en.wikipedia.org/wiki/Valentin_de_Foronda}{Valentín de Foronda}\footnote{Consul general en Philadelphia y Ministro plenipontenciario en los Estados Unidos.} quería hacer una España dividida en cuadrados de 500 leguas cuadradas cada uno, con números en vez de nombres para borrar cualquier rasgo de historia de cualquier sitio, para que todos nos fuéramos de la provincia 1 2 3 4 y los afrancesados quisieron hacer provincias fluviales también en España con nombres de ríos para acabar con los viejos nombres porque se trata de crear la lealtad. Y en Francia lo hicieron adrede cogieron los viejos reinos, los cuadricularon para aguarlos, para por ejemplo, en Euskadi combinaron en los departamentos territorios vascófonos que hablaban Vasco con territorios francófonos para aguarlos y diluirlos, los metieron todos en la misma provincia y se trata de crear de forma artificial unidades administrativas nuevas sin ningún tipo de rareza histórica, sin ningún tipo de lealtad, de tal forma que los ciudadanos solo obedecieran al centro. Y esto no solo se centraliza por ahí, se centraliza, por ejemplo, queriendo cambiar con el idioma, con las leyes, creando una cámara sola, un parlamento solo, la idea una cámara única es una idea también muy republicana para que no haya digamos dos parlamentos en contraposición o no haya parlamentos de base territorial sino que haya  un parlamento que represente supuestamente a toda la nación francesa y que no haya ningún tipo de lealtades. Se ve, por ejemplo, con la implantación del sistema métrico decimal, que es un invento de la Revolución francesa, se trata de hacer todo en base 10, de hacer todo de base racional, querían hacer las horas de 100 minuto, querían hacer días de 10 horas, querían hacer todo en base decimal, cambiaron el calendario, cambiaron un calendario republicano e hicieron semanas de 10 días también, supongo que lo saben, querían hacerlo todo así como muy métrico, olvidándose todo el pasado, rehaciendo el pasado, borrando la historia y creando digamos como un mundo nuevo totalmente racional y totalmente geométrico. Quedan restos, si se fijan muchos países africanos, sobre todo los de origen francesa, tienen diseño geométrico, geométrico cuadriculado de escuadra y cartabón, es un residuo de querer hacer países racionales, cuadrados o con formas geométricas puras, es un ideal.  si no miren los mapas del Sudán o miren los mapas así a ver si digo verdad o mentira, son residuos de esta idea decimal. Y por supuesto estandarizan los pesos y las medidas, yo no sé aquí, yo en mi tierra los gallegos antiguos aún contaban encerrados, contaban en ollas, en bollos, en medidas estándares que variaban según las zonas. Ya nos puso una cosa fea, el kilo y el litro. se obligaron y españolizaron a poner los litros y kilos, supuestamente para estandarizar. 

Si leen el libro de \href{https://en.wikipedia.org/wiki/James_C._Scott}{Scott}\footnote{Fue un político y antropólogo estadounidense especializado en política comparada.}, un viejo anarquista, se llama ``Seeing like a State``\cite{scott1998seeing} explica que estas cosas no son tan inocentes. Se trata de que los pesos y las medidas sean fáciles de medir y los metros cuadrados para que los Estados puedan cobrar tributos. Si cada zona tiene una denominación distinta el estado central no puede saber cuánto mide, por ejemplo, una olla de mi zona, no es lo mismo que una olla de Rivadavia, son distintas las ollas, el cerrado, el tal, no son las medidas exactamente iguales, así es difícil de medir. Pero el Estado estandariza las medidas, usted piensa en que la geometría le inventan los sacerdotes, las matemáticas que estudian ustedes y buenas partes de ella se inventan los sacerdotes egipcios para para cobrar tributos en su momento porque porque es más fácil calcular superficies áreas y medir producciones si tenemos medidas estándar si yo produzco 1.000 ollas si el otro produce 1.000 cavazos cómo medimos eso, no es fácil para un gobierno hacerlo. Entonces claro hay que tener una medida estándar para nuestro bien, para eliminar fronteras y cosas por el estilo. También los Estados nos ponen nombres y apellidos y direcciones. ¿Por qué ustedes tienen un nombre y dos apellidos? Porque ustedes en la vida real no serían Juan López Fernández o Miguel Bastos Boubeta no, yo sería Miguel o filho da señora María o la señora Nieves, así me conocen allí, pero el estado así no me localizaría. ¿Cómo localizaría el Estado central para reclutarme, para pagar tributos, como o filho de la señora Nieves no me localizarían así, necesitan un nombre, apellido y un número al lado para teclearlo y salga yo rápido. Son procesos de estandarización que facilitan mucho la labor de gobierno y son parte del proceso de centralización administrativo de la Revolución francesa. Pero cuidado esa revolución, ese proceso de centralización administrativo fue imitado aquí y fue imitado sobre todo en Italia, creo que pocas administraciones resistieron este influjo, entre ellas la suiza, que es mi administración favorita porque no tiene ningún tipo de lógica de este estilo. Pero aquí siempre se imitó, ¿Quiénes? Los liberales, van a ver porque yo les critico que prefiero los viejos carlistas independentistas y a los viejos monárquicos que a los liberales.

Los Liberales copiaron \href{https://en.wikipedia.org/wiki/Spanish_Constitution_of_1812}{la constitución de la Pepa}, esa que tanto le gustan así a los liberales, es una constitución que es la primera vez que se habla de nación española, antes ese concepto es absolutamente ajeno a la tradición política española, eso lo copian de la constitución Jacobina también que hablan de la de la nación francesa. Entonces los liberales hispanos copian esas cosas y también hacen provincias después y hacen este tipo de esquemas de centralización administrativa para hacer nacional las provincias y todo eso. En España nos salvó que no hicieron las provincias como en Francia al disolver los reinos sino que más o menos aún conservaron los viejos reinos y dividieron los reinos en provincias porque la idea es hacer provincias a dividir también, o sea, coger, por ejemplo, parte de Lleida mezclarla con parte de Aragón y hacer una provincia con eso para que no hubiera mayoría de nada y aquello irlo aguando poco a poco para que no hubiera digamos territorios más o menos homogéneos. Aquí se intentó pero al final eso no cuajó, pero la idea está copiada de Francia, la idea de centralización está copiada de allí. Incluso se uniformiza el vestido, y el trato, las modas de hablar a la gente de tú y así, eso viene de la Revolución francesa, el idioma siempre tiene una serie de grados de respeto y son de respeto no es de humillación, sí es de respeto, es decir, yo cuando a ustedes les hablo de usted porque yo los respeto a ustedes, no digo \enquote{eh colega, ¿Qué opináis?, ¿De qué vas?} y tal, no hablo así, estoy hablando en sede académica, les trato con el respeto que merecen y les llamo de usted, etc. si hay amistad pues ya relajaremos el trato, ya nos llamaremos de tú, otro tipo de cosas. Pero al principio como yo respeto a las personas, como ustedes que saben más que yo, les hablo de esto. Entonces, digo, se relaja el trato, se relajan las formas adrede precisamente para crear individuos indiferenciados, individuos masa que puedan ser fácilmente gobernados, se trata de no crear lealtades, ni grupos, ni colectivos que se diferencien de una forma de la otra y se uniformiza el vestido sobre todo el traje de caballero. El traje de caballero se fija en la ropa del Antiguo Régimen, el caballero iba vestido con \href{https://en.wikipedia.org/wiki/Livery}{libreas}\footnote{Librea es el diseño identificativo, como un uniforme, ornamento o insignia, que indica propiedad o afiliación, normalmente utilizado en personas o vehículos.}, tenía \href{https://es.wikipedia.org/wiki/Gorguera}{gorgueras}\footnote{La gorguera es una pieza indumentaria a modo de pañuelo fino ya en desuso que cubría el cuello o el escote.}, iba con ropa de colores, iban así con pelucas y cosas por el estilo. Para igualar, la Revolución francesa circunscribió el uso de tres colores a los trajes de caballero, de otra forma que los caballeros a día de hoy se lo pueden llevar trajes gris, negros o azul marino, eso fue una cosa impuesta por la Revolución francesa para igualar también para que la gente no presumiese. Esto es real, pueden pueden contrastarlo también si quieren. Hay una idea de centralización, de concentrar cada vez más el poder en manos del poder central y se trata siempre de que las personas sean leales al estado central y no sean leales a su territorio, a su etnia, a su lengua o a su cultura y eso se intentó hacer. Y en Francia funcionó, esperemos que aquí no funcione porque el proceso en España fue exactamente el mismo, copiado con menor eficacia y con menor fuerza del centro, pero básicamente es el mismo y lo consiguió en buena medida.

Tres, laicismo. Otro de los grandes enemigos de cualquier poder central es la existencia de religiones más o menos organizadas porque la religión presenta una moral autónoma distinta de la moral del gobierno y el gobierno pretende ser él la fuente de la ley y la fuente de la moral. El estado no existe, son personas, entonces las personas que gobiernan quieren ser ellas las fuentes de la moral y la fuente de la legislación, es decir, ser ellas las que deciden lo que está bien y lo que está mal y desde hoy, de hecho, hablamos de una cosa que es legal o ilegal no si es buena o mala y puede haber cosas malas legales y cosas buenas ilegales y eso se establece a través de crear una moral propia y autónoma del estado diferenciada de cualquier tipo de moral religiosa. Por tanto siempre los principales enemigos de los gobiernos modernos son las religiones organizadas y la Iglesia católica es la iglesia ideal para para meterse con ella. Como antes dije la Revolución francesa no es el origen de todos los males, muchos de los males ya venían de antes, la centralización por ejemplo, en Francia, ya venía de antes venía del Rey Luis XIV que empezó tímidamente. En España empezó por el nefasto rey Felipe V que puso los decretos de nueva planta y le quitó los suelos a los catalanes y les hizo muchas servicias y perrerías. Esos procesos ya venían de antes, como buenos Borbones tienen una tradición de centralización, pero el proceso llega a su paroxismo en la Revolución francesa. Los procesos de expropiación de las iglesias y de las religiones organizadas son antiguos, todas las reformas protestantes de Europa lo que básicamente es expropiar las tierras de la Iglesia, quitárselas, repartirlas entre los nobles o entre los amigotes y crear una suerte de religión estatal obediente. Uustedes miren, por ejemplo, el reino de Inglaterra es la cabeza de la iglesia de Inglaterra, la reina de Noruega lo mismo, la reina de Dinamarca lo mismo, etc. son las cabezas de sus respectivas Iglesias porque se crea una cosa tan absurda como Iglesias estatales, o sea, cómo va a ser los que son noruegos se salvan los demás no, es una cosa un poco así peculiar. Pero crearon ideas nacionales para expropiar las tierras y para tener una fuente de legitimación religiosa al servicio del poder político la Iglesia católica normalmente es más dura de roer y la atacaron también con lo mismo, con las mismas cosas, primero creando una iglesia estatal, la llamada Iglesia constitucional, que obligaba a los curas a los sacerdotes a jurar lealtad a la constitución republicana, la mayor parte de los sacerdotes se negaron a jurar la Constitución y se convirtieron en curas refractarios, que se llamaba así, se negaron a ocupar eso. Entonces se atacó severamente, se expropiaron las tierras, muchos de los curas refractarios se los persiguió con dureza y se puso una especie de curas constitucionales del Estado en su lugar. Y además, primero, se atacó duramente el cristianismo y después \href{https://en.wikipedia.org/wiki/Maximilien_Robespierre}{Robespierre}\footnote{Líder prominente de la Revolución Francesa.}, que era más astuto, quiso crear una religión de estado y creó con su amigo \href{https://en.wikipedia.org/wiki/Jacques-Louis_David}{David}\footnote{Pintor francés de estilo neoclásico.}, David como lo saben es el pintor de la revolución, pero además del pintor de la revolución era el que organizaba las grandes performance de la revolución, o sea, era un gran organizador de espectáculos, de espectáculos cívicos, montó, por ejemplo, el entierro de \href{https://en.wikipedia.org/wiki/Jean-Paul_Marat}{Marat}\footnote{Teórico político, físico y científico francés asesinado durante la Revolución Francesa.}, el entierro de \href{https://en.wikipedia.org/wiki/Louis-Michel_le_Peletier,_marquis_de_Saint-Fargeau}{le Peletier}{Político francés activo durante la Revolución Francesa.}, montó, por ejemplo, el llamado culto a la razón que pusieron allí unos actrices allí en Notre Dame de París prestándole culto a la razón, pero eran  espectáculos auténticamente de masas, con coros, con liturgia, o sea, la gente vestida con cirios y todo para hacer una especie de ritos a la razón, una especie de culto al Dios razón porque de Robespierre rápidamente se dio cuenta que necesitaba algún tipo de culto religioso para dominar. Imitando lo que hacían los romanos, imitando lo que dice \href{https://en.wikipedia.org/wiki/Niccol%C3%B2_Machiavelli}{Maquiavelo}\footnote{Fue un diplomático, autor, filósofo político y escritor italiano, considerado el padre de la filosofía política moderna y de la ciencia política.}, en discursos de la primera década de \href{https://en.wikipedia.org/wiki/Livy}{Tito Livio}\footnote{Fue un historiador romano que escribió una monumental historia del Estado romano en ciento cuarenta y dos libros (el Ab urbe condita), desde la legendaria llegada de Eneas a las costas del Lacio hasta la muerte del cuestor y pretor Druso el Mayor.}, dice cualquier estado tiene que crear algún tipo de religión civil que sustituya la religión religiosa con los mismos ritos, con los mismos fastos, pero que le rinda culto, en vez de a una divinidad, que le rinda culto al Estado. Entonces, se crea, por ejemplo, la plantación de árboles de la libertad, entonces hacían mucha fiesta y plantaban árboles de libertad que los contrarevolucionarios los cortaban después, no les gustaba aquellos árboles de la revolución, era lo primero que cortaban cuando cuando hacían revueltas. Entonces crearon ritos y crearon un principio de religión cívica con muchos cultos y muchos fastos, de eso se trata la idea de sustituir lentamente la lealtad a la religión, sea esto cual sea, por una religión estatal, de tal forma que la gente en vez de dar la vida o sacrificarse por religión, dé la vida o se sacrifique por la República, por la Patria, por la Nación o por el Estado y normalmente se sacrifica por las personas que gobiernan eso, que es un proceso bastante bien conseguido. Y repito, David era un mago en hacer eso, un mago en hacer cuadros, el famoso cuadro \href{https://en.wikipedia.org/wiki/The_Tennis_Court_Oath_(David)}{``el juramento del juego de pelota``} pues veis un cuadro representativo de La Revolución Frances, cuando ustedes ven ese cuadro, pues David era el pintor y escenógrafo de todas aquellas cosas, con culto muy bien montado. Entonces ellos instauraron el culto por toda la República y días de fiesta porque cambiaron los festivos, cambiaron los calendarios, para adecuarlos a los nuevos cultos cívicos y nacionales para crear una especie de religión del Estado.

La Iglesia en aquel momento era una iglesia muy poderosa y aparte era un enemigo tenía sus propios sistemas tributarios
tenía sus propios sistemas de de de justicia entonces era un enemigo terrible intentaron
todas las formas debilitar y le hicieron mucho daño los sacerdotes católicos fueron muy reprimidos
sobre todo en las en las grandes revueltas de la que veremos después de la bende o en las revueltas de Bretaña o en en otros sitios
los sacerdotes eran los que lideraban al pueblo y como tanto eran enemigos del del poder político y poco a poco
lentamente fueron siendo fueron siendo atacados otro aspecto
El "Culto a la Igualdad"
que puede ser criticado de la Revolución francesa es su
culto al igualitarismo su culto a la igualdad es uno de los lemas de igualdad fraternidad y libertad
la idea de igualdad el culto al hombre medio el culto a la al al al individuo que nos destaca
el el culto al al al individuo igual eran tan iguales los jacobinos que eran los paladines de todo esto
que hasta tiraban las torres que eran más altas creía que todas las Torres en Estrasburgo llegaban a tirar la torre del reloj de la catedral
porque era más alta que el resto quería que todos los edificios fueran igual si instaura una especie de culto a la igualdad
sobre todo de la igualdad de oportunidades y cómo se concreta señores igualdad de oportunidades
La Educación Pública, una herramienta del Estado
ajá con la escuela estatal obligatoria que supuestamente es el mecanismo de ascenso social no
esto empezó con la revolución a quien lo desarrolló después fueron los fueron después dulce río otros
pero la idea que ya está en Marat o ya está en robbisppier de crear creo que está en robbispier también de crear una escuela pública
una instrucción como le llamamos obligatoria está ahí también para qué para garantizar la igualdad
de oportunidades entonces con esa excusa que hace un sistema básicamente que consiste en educar a los niños
repito la la revolución industrial comienza a hablar de esto aún no lo instaura no lo instaura lo instaura lo instaura después
pero se está en muchos principios que que que bien después de la de la revolución industrial la idea de crear un sistema de igualdad
de oportunidades la idea de crear hombre después veremos que los la conspiración
que hubo después de la caída de los jacobinos en 1 795 los jacobinos cayeron en 90 y
cayeron en 93 internidor del 93 pero después la hubo una revuelta
"La Revuelta de Los Iguales" (1796); François Babeuf
la llamada la revuelta de los iguales uno de los primeros comunistas modernos que fue grákova beff
que intentó hacer una revuelta comunista también la era la conspiración de los iguales querían llevar los ideales jacobinos al último extremo
y querían llegar a una especie de revolución comunista en el que todos fuéramos iguales era una especie de culto a la igualdad y el culto a la igualdad
se expresaba en lo que dije antes en el vestido en las formas y después en la instauración de sistemas de educación
o nivelación social de la forma que todo el mundo educara a la misma escuela se educaran leyeran los mismos libros
y y aprendieran los mismos valores y sobre todo si miran el contenido de sus valores ven que sus valores son casi todos ellos
valores estadistas valores de básicamente de dominio del
del de obediencia al estado porque la básicamente la instrucción pública el tal como la planteaban
La Educación Pública forma ciudadanos obedientes
era de crear ciudadanos obedientes ciudadanos que obedeciesen al estado y que diesen su vida por él
que pagasen tributos por él y eso lo consiguieron en buena medida lentamente prueba está en
La "Guerra en masa", creación de la Revolución Francesa
en la guerra en masa es otro punto otro otro gran invento de la Revolución francesa cuál fue la leva en masa
el servicio militar obligatorio gran logro el ejército de masas
lo que llamaba el viejo nickerson en su libro la horda armada tiene por por primera vez
"The Armed Horde" (1940) - Hoffman Nickerson
en vez de ejércitos de pequeña escala ejércitos de Antiguo Régimen se plantean ejércitos a gran escala
ejércitos de millones de hombres en balmi por ejemplo frente a las frente a las tropas austriacas
le planta un ejército de escenas de miles de personas que arrasan por el mero número pero eso implica levas en masa
implica por ejemplo que buena parte de la población campesina tiene que dedicar uno o dos de sus hijos a pelear por las guerras del
de la de la patria francesa entonces se crea ese invento nuevo fue un invento diabólico porque transformó diabólico
porque transformó las guerras del Antiguo Régimen que eran guerras o hombres no que que fueran buenas pero eran guerras digamos a pequeña escala
la sociedad más o menos tenía guerras casi continuas o mucha frecuencia pero la población en general no se veía afectada
es como las guerras que tenemos ahora pues hay una guerra en Irak o en Siria y ahí hay tropas españolas allí
pero nosotros no nos vemos muy afectados porque son pequeños contingentes de tropas que pelean en otros sitios
y se ven afectados los de allí pero nosotros no esto transforma la guerra de tal forma
La transformación y deshumanización de la guerra
que después de bien en las guerras napoleónicas guerras que causan en Europa millones de muertos que es algo casi
relativamente poco visto en tan poco tiempo y transformar las grandes carnicerías demás es la primera y Segunda Guerra Mundial
en que en que son guerras de pueblos contra pueblo no guerras de ejércitos contra ejércitos como eran las guerras señoriales
que peleaban un ejército normalmente de mercenarios contra otros agentes mercenarios pero no eran guerras en las cuales el pueblo estuviera involucrado
de hecho si se fijan en las guerras del Antiguo Régimen aún habiendo guerra había comercio entre los dos pueblos en guerra ahora se transforman las guerras
en guerras de todos contra todos ahora todos los franceses son mis enemigos todos los ingleses son mis enemigos
y hay inglés que pille inglés hay que matar inglés Alemán o lo que sea se transforma es una en una deshumanización de la guerra
en una carnicería eh el individuo pierde su individualidad y se transforma en uno más en una masa eso se transforma una guerra de masas
y esto se debe a la Revolución francesa a ese invento de de que es el a las armas ciudadanos
formar los batallones no esas cosas que están en la en la Marsellesa dice que todos a la guerra
todos a pelear todos a defender la pata el problema es que se defendieron los otros también con ejércitos de masas
y al y al final qué pasó que tocarán guerras a pequeñas caras transforman en guerras a gran escala
y no solo eso lo francés obtuvo también otra cosa la idea de exportar la revolución
La idea de "Exportar la Revolución" (Jacques-Pierre Brissot)
los la Revolución francesa tuvo también sus em neocón
partidarios de hacer guerras en el exterior para para para quitar problemas dentro la Revolución francesa fue tremendamente expansiva
y no fueron los jacobinos fueron los girondinos que supuestamente era la derecha
de la revolución eran los más así como lo así como sí como lo más centro derecha son así
los neoko no y eso eran los belicosos que pasa que los jacobinos después apuntaron rápidamente
también a la cosa pero la idea de de esto era de los de brishot básicamente eran de los de los girondinos
que es que son los que empiezan este tipo de guerras de expandir la revolución para afuera ahora se habla de expandir la democracia
entonces Estados Unidos hace guerras por todo el mundo para expandir la democracia para expandir la
la los valores de virtud ese concepto lo explica un señor que se llama klais Rin
"America the Virtuous" (2003) - Claes G. Ryn
uno que se llama América de virtuos la América de la virtuos explicando como muchos de los conceptos de los neocones sacan de la idea de la Revolución francesa
de expandir la revolución y liberar al resto del mundo Robert Spears tenía una frase
"Hay que obligar a la gente a ser libres" - Robespierre
que que después mucha gente mucha gente se apuntó ella que decía que hay que obligar a la gente a ser libres
hay que usar la fuerza para que la gente sea libre entonces se trata de invadir otros pueblos
para um quitarles las cadenas y que sean ellos libres y para eso es establece unas guerras a gran escala
que después devienen en las guerras eh napoleónicas Napoleón básicamente es un sucesor
Napoleón, el sucesor de la Revolución Francesa
de la Revolución francesa Napoleón cambia algunas cosas pero básicamente el código napoleónico es una
pervivencia a la Revolución francesa los los ideales de Napoleón tal cual los cogían los afrancesados de aquí
eran los ideales de la Revolución francesa se trata de expandir la libertad por la fuerza entonces se desatan las furias
como diría Arno Mayer por toda Europa y sí se se se estaban en guerras
revueltas y revoluciones por toda Europa causando entre esto y la guerra polémica varios millones de muertos saqueando Europa y haciendo
un daño tremendo que llega hasta aquí la guerra llega a España llega a Cataluña llega a Portugal llega a todas partes se expande por toda Europa
porque esto es otra cosa los amigos de la nación francesa pues Napoleón es el primer intento de unificación política
Napoleón, el primer intento de unificación política de Europa
de Europa hubo tres así recientes Napoleón Hilder y ahora la la Unión Europea
hombre no los comparo pero a mí el beresi no los comparo eh no es lo mismo
pero digo que el beresi también me parece bastante bien como buen independentista que soy
porque es un intento de unificar por la fuerza Europa Europa el genio de Europa es estar rota
El genio de Europa es estar fragmentada
estar fragmentada ese es el genio de Europa es el triunfo de Europa el nunca estar el estar Unidos culturalmente
pero divididos políticamente ese es el éxito ojalá España fuera así rota políticamente y unida culturalmente
emocionalmente por otras cosas pero rota políticamente porque es ese es el ideal de Europa
no estar nunca unida a diferencia de China que estuvo siempre unida desde hace 1 000 y pico de años o 2 000 años a ella sí le fue
Europa su éxito está en ser roto y eso lo quiso romper la Revolución francesa también quiso crear una especie de espacio europeo
así de de libertades y así normalmente bajo la égida del de los franceses
más de Napoleón que de la que de revolución pero digo esa idea mesiánica de querer expandir la revolución a todas partes
Los Marxistas admiran la Revolución Francesa
querer expandir la democracia que después también tomaron los trotskistas porque todos quieren un gran admirador de la Revolución francesa a los marxistas les gusta mucho la idea de revolución
porque es el modelo que ellos quieren un golpe de efecto una toma de poder acabar con la vieja élite instalar una élite nueva
e instaurar un nuevo régimen político entonces la el mito de la revolución pervivió mucho en el tiempo y un mito fue un mito bastante violento
La Economía Francesa durante su Revolución
otro punto más es el punto de la economía la economía la economía francesa
muchas veces se vende la Revolución francesa como una revolución burguesa al principio lo fue a ver la la economía francesa antes de la población
estaba relativamente prosperando tenía unos problemas de déficit fiscal en los años anteriores por eso se convoca a los Estados generales
porque había problemas de déficit esta el estado francés se metió en las guerras americanas en la guerra de independencia de Estados Unidos
se metió allí debían dinero después no tenían no tenían no tenían con qué pagarlo vinieron las malas cosechas y
y quisieron recaudar impuestos convocando los Estados que es como se hacían ante las cosas no podían subir impuestos
sin convocar ante los parlamentos y a los Estados les dieran autorización para poner tributos
entonces quiso recaudar dinero más dinero de la iglesia o más dinero de lo de los nobles o lo que fuera y convocó al estado que es divino en todo eso
pero una economía relativamente próspera estaba creciendo tanto como la inglesa y no desmerecía nada a otras potencias europeas
o sea con la monarquía no fue aquello para abajo no fue basta iba bastante bien
era un país relativamente próspero puertos como el de Burdeos etc. eran puertos muy muy próspero no eran país
no era un país digamos para nada atrasado Europa y al principio la revolución tomó un tinte de libre mercado
Jean-Baptiste Say
piensen que uno de los economistas clave de la Revolución francesa es yambasticsey que colaboró uno de los padres de la escuela austriaca
colaboró en casi todos los los puntos de colaboró mucho en la en los pesos de la revolución lo que pasa con la revolución después se desmandó
al principio sí hicieron unas propuestas de ir de mercado pusieron propuestas por ejemplo de eliminar
el proteccionismo al al grano de de de eliminar algunas barreras al barreras al comercio
eliminaron con la la ley hecha peleé eliminaron los gremios hicieron una cosa al principio muy con una retórica
muy libre mercado que rápidamente quedó aquello nada que en cuanto empezaron a hacer de las suyas en cuanto expropiaron las tierras de la iglesia
y empezaron a meterse en guerras y empezaron a meterse en cosas empezaron a a crear 1 1 serie de bonos o de deuda
llamado asignados con el respaldo de la tierra de la iglesia pero asignados que después
empezaron a emitir a gran escala nadie los quería los declararon de curso forzoso y crearon grandes infracciones
eso ya lo había hecho el Rey Carlos 5º en España eh con los juros eso de de obligar al curso forzoso de la deuda
ya no es una cosa ya no es una cosa nueva pero digo que ellos lo llevaron a gran escala llevaron 1 a 1 inflación de caballo
Los decretos de "Maximus" (Albert Soboul, Henry Hazlitt)
como llevan una inflación de caballo que es lo que hacen inmediatamente después los gobernantes decretos de máximo para que para que no suban los precios
entre que había malas cosechas que había guerras y que y que había una expansión monetaria de caballo normal que los precios de las de los consumos suban
eso lo explica cualquiera que sepa un poco de de economía que para que los historiadores están embuidos de digamos de conceptos económicos de otras escuelas
y eso no no lo acaban de ver lo ven todo por ejemplo Soul o historiadores marxistas lo ven como en en términos de lucha de clases
de acaparadores de especuladores quieren robar al pueblo yo digo es del libro si usted emite más dinero
suben los precios y pone un control de decreto de precios hay escasez no hay comida en las tiendas lo explica Henry jazz
y por ejemplo ya es un librito de la economía en buena elección esto es de libros de primero decir es una
es una cosa fácil de de entender por por muchos no los entienden pusieron todos los eh errores económicos
al uso máximos eh inflación etc. y causaron un desajuste monetario a gran escala
que causó revueltas que los gobernantes como ahora hacen en Venezuela
astutamente echan culpa a acaparadores especuladores todo tipo de agiotistas todo tipo de gente mala
que son la que roba al pueblo y ellos están allí para defender al pueblo y otra cosa más quiero criticar de la Revolución
La idea de "La función social de la propiedad" (Jean-Paul Marat)
francesa en este aspecto la idea de la función social de propiedad
es una cosa que ya marata habla de ella y está en muchas constituciones entre ella la española de ahora para las construcciones liberales de siglo 19
y todos los sucesores de la Revolución francesa copiaron esa maldita idea la idea de que la la propiedad no es tuya
es una tiene que tener una función social y quién determina la función social ellos Marati compañía
entonces que la propiedad ya no es suya la propiedad se la pueden el uso de ella ya no lo determinan ustedes
lo determinan ellos y su propiedad es nuestra podemos expropiarla para el uso que nosotros necesitamos pertinente
expropiar tanto su tierra como el uso de su tierra a lo mejor su tierra dice no su tierra está mal aprovechada porque está usted plantando maíz
y tiene usted que plantar trigo entonces tiene usted que plantar trigo en su tierra ya le expropian el uso porque el uso social es que planta usted kiwi no maíz
o que use su tierra para la industria y no para tenerla quieta entonces ya se establece un principio nuevo que Mina por completo los derechos de propiedad
a mí me gusta decir por ejemplo que a día de hoy en España con esta legislación que hay ahora que deriva de la Revolución francesa
En España no hay Propiedad, hay Usufructo
esta idea de la función social en España no hay propiedad hay usufructo usted su tierra no es suya
igual que su renta no es suya porque el gobernante decide cuánto le quita en cada momento y dice quiero quitar 70 por 100
se lo quito le llamo impuestos o progresivos lo que sea pero le quito 70 por 100 le quito 75 al 80 al 90 no en límite lo quito yo lo que quiera
y con su propiedad lo mismo se la le cambio el uso usted en su tierra quiere puedo decir no
su tierra es rústica digo yo que rústica o digo yo que su tierra es industrial o digo yo que su tierra es urbana
si quiero hacer un favor su tierra no es suya el el uso se lo doy yo y el uso que me apetece a mí paso yo el soberano y el
el representante de la voluntad general como diría ruso que es otro gran otra gran palabrería esto de la voluntad general
interés general entre cosas todas son palabrería de la Revolución francesa en el nombre del cual se ejerce el
se ejerce el dominio pero es esa idea de la propiedad esa idea del de quitar despropiar campos liberales
Los Liberales fueron los grandes expropiadores en España
fueron los grandes hispanos y dignos herederos de la unión industrial de la Revolución francesa perdón son los grandes expropiadores de tierra
en España son los que expropian los comunes todos los que expropian las tierras a las comunidades los que les propian las tierras a las universidades
los que les propia las tierras a la iglesia y las reparten entre sus amigotes primero los francesados
con Pepe botella y después los liberales de Mendizábal son todos los los que es propio y no digo mentiras
son es lo que es propio la tierra porque la tierra tiene una función social entonces ellos la expropian el nombre del interés general
y después la reparten entre sus amigotes como hicieron primero los afrancesados que saquearon buena parte del arte español
aprovechando los franceses lo lo saquearon y lo vendieron los afrancesados
qué es esa idea despropiar de quitar la tierra de de eso lo decía burk
también que hablábamos antes de él burk decía que hay que defender no solo la pequeña propiedad sino la gran propiedad
y la gran propiedad es la primera que va a expropiar porque estás en defensa quiero recordar por último
"El Terror", el legado de la Revolución Francesa
por último el terror
el terror no es algo nuevo tampoco en la en la política los griegos lo usaban a gran escala
no más a gran escala pero el terror políticos revive con la Revolución francesa
La diferencia entre el "Miedo" y el "Terror" Político
distingamos el miedo del terror el miedo normalmente se refiere a un grupo a un colectivo concreto por ejemplo yo
si yo la ganara yo qué sé a mis si mis amigos comunistas siempre dicen je je je
que cuando gane allí la marea o es una broma eh que no yo los quiero no que me quieren mucho siempre decía que me iban a poner a a picar piedra
en las canteras de Porriño cuando je en las canteras que hay allí en Galicia cuando ganen ellos que tienen una pica
una pala para mí pero bueno yo lo entiendo es una broma eh que no no
no no hay remedios de de criticar no en en serio em es una broma te digo
es normal si llegaran a un régimen de izquierdas o muy de izquierdas muy radical yo tuviera miedo
que yo también las busco no sé si me explico también las busco yo
si llegara un régimen así tipo revolución bolchevico así es normal yo las busco yo soy una persona que no está callada
y soy así yo como soy entonces normal que yo tenga miedo pero el miedo estaría circunscrito a personas como yo
que están metidas en política o están metidas en discursos políticos están metidos en colectivos que digamos que se enfrentan políticamente al nuevo régimen
pero el terror no es eso el terror se puede ser cualquiera usted usted usted y yo no entiendo nada pues igual
ahorcado fusilado para para poner respeto es el terror que no sepa no sepa dónde va a venir el miedo es cuando se sabe
los que saben saben que hicieron algo y saben qué les puede venir saben lo que están haciendo entonces digamos el miedo está circunscrito
y entonces el miedo en un principio afecta solo a los que están metidos en en luchas políticas o en cosas
pero el terror es cuando puede ser cualquiera y el y eh y se aplica como arma precisamente para que incluso
la gente delate a sus familiares delate a sus amigos no tenga ningún tipo de lealtad por el terror que tiene porque puede ser cualquiera
y se usa para debilitar los lazos sociales para que el dominio sea absoluto y se usa
el terror se usa de forma estratégica de forma estratégica tanto la Revolución francesa como la revolución rusa
como general cualquier revolución dura que se especie mete terror no miedo miedo es concreto se sabe quién es
terror es cualquiera y se usa para debilitar y ellos usan el terror a gran escala
"Noventa y tres" (1874) - Victor Hugo; "El Reinado del Terror" (2008) - Robert Margerit
usan el terror en las en sobre todo el año 93 yo recomiendo la la novela de Víctor Hugo 93
que es una novela excelente que que explica muy bien el el clima en aquella época o las novelas de Roger marger Hill
por ejemplo el el reinado del terror que explica muy bien el el el tiempo cualquiera podía ser cualquier denuncia
su vecino cualquiera usted no aquel aquel de allí es es contrarrevolucionario y ya inmediatamente
lo lo podían ejecutar y se usó durante una temporada para meter miedo es una es una estrategia política no y y se usó
y su sol el terror a gran escala en las represiones a gran escala en Francia hubo varias rebeliones
hubo en el sudeste en en en Marsella y en y en león que fueron duramente prestadas
Guerra de la Vendée (1793 -1796) (Jean Chouan)
pero hubo una guerra civil a gran escala que algunos historiadores califican de genocidio que fuera el levantamiento de la bende
no sé si tengo que hablar del levantamiento de la bende y todas las guerras de las de los chuanes de la chubanería del mítico Juan Chuan
que era 1 1 ser mítico que luchaba por sus pueblos que normalmente las regiones conservadoras de
de de Francia ellos decían los revoluciones francés decían la reacción
los reaccionarios hablan Vasco decía por las razones en nombre de su lengua en nombre de su idioma
en nombre de su cultura en nombre de su tradición de sus leyendas algo así se rebelaban contra la nueva política de París
o en nombre de religión o de sus santos o de sus cosas así se levantaron en varios religiones de Francia
contra eh contra el poder central y hubo una represión brutal algunos historiadores la califican de genocidio
es otro de los debates historiográficos en los que no me quiero meter porque no soy especialista pero sé que hay de hay historiadores que dicen que fue un genocidio
otros como arnomayo lo matizan la shot por ejemplo dice que es un genocidio Arnold Mayer dice que lo matiza
que fue un contexto de guerra civil que hay que verlo de otra forma no me quiero meter en eso pero sí que todos coinciden tanto unos como otros
que fue una represión a muy amplia escala entonces ese terror
fue el último de los legados de la Revolución francesa porque ese terror se aprendió y se usó en revoluciones posteriores
porque es es fue un arma la Revolución francesa fue muy estudiada después por otros revolucionarios la revolución por excelencia
entonces todo cualquier revolucionario que se preciaba estudiaba muy bien la Revolución francesa y aplicaba las técnicas que usaron que usaba el directorio
que usaba el comité de salud pública para gobernar y para no cometer los fallos que ellos cometían
El Tiranicidio (Charlotte Corday)
eso sí también la Revolución francesa también tuvo su tiramisillo no quería marchar aquí si recordara a charrot cordell no sé si alguien la conoce
recordé y fue fue la una joven así de su edad que después de leer clásicos como Plutarco
y dicen algunos autores yo recuerdo haber leído que leyó algún clásico del tiranifirio del estilo de de no no di no dije a Juan de Mariana
pero del estilo de Juan de Mariana del estilo de de el tiranicidio de la doctrina que es el Rey
abusa de su poder puede ser muerto por cualquier súbdito es una doctrina del tiramicidio es una doctrina de origen medieval
Vindiciae contra Tyrannos" (1759) - "Philippe de Mornay
que está en dupress morné en en la Vin diquia contra tiranos y está una serie de autores sobre todo pues jesuitas y teóricos de escolásticos
que defienden que si el Rey abusa cualquier ciudadano tiene el deber o el derecho de darle muerte
pues una esta señora una joven con ustedes influida por putarco seguro seguramente por la por bruto
que es otro de los viejos referentes de esto bruto el que asesina a Julio César antes que se vuelva muy poderoso
Carlota cordal asesina a marath en su en su baño
y si convierte inmediatamente en en una heroína de lo de la contrarrevolución y durante mucho tiempo muchos sectores en Francia la divinizaron
precisamente por aplicar la eternidad de de de del tiranicidio porque Marat era un monstruo
no digo yo dicen los historiadores era fue que asesinó a los presos indefensos en la en las cárceles de París
o sea era una persona absolutamente represor de hecho Carlota Carlota Maray lo lo
lo asesina cuando ella ella les da unos nombres y dice inmediatamente estos nombres van a ser ejecutados entonces es cuando ella lo
lo asesina asesina al a a uno de los principales artífices y de los más crueles de la revolución
cosa humana entonces tiene su teoría del del tiramicidio también bueno
Conclusiones
yo con esto quería invitaros ya sin aburrirlos más quería invitaros a que estudiaran estos temas para no cometer así los errores que cometo yo
o no hablar de forma tan simple como puedo hablar yo yo repito no soy historiador me interesa el tema
pero no soy un profesional me gustaría eso sí que los profesionales de verdad que incorporaran a sus estudios
a su riqueza a su audición algún elemento de este estilo si están de acuerdo con él si no pues aquí estamos para discutirlo muchas gracias
quién gobierna aquí soy yo las palabras
que has dicho que en la revolución van a catalán sus plau eh yo os tomo el tuyo el catalán ah Messi igual yo no
yo no quiero venir aquí a hacer esa habla castellano lo que faltaba no no da igual en
yo no hablo castellano nunca yo por eso se lo digo no es ábrelo como quieras que en la Revolución francesa
Pregunta sobre el Sistema Métrico Decimal
has dicho que intentaban imponer el sistema decimal sí les pusieron allí
no no sería más bien que evolutivamente
em la gente hubiera aprendido a contar con los dedos que son 10 y a partir de ahí implantar el sistema decimal
hay varios hay varios patrones usted en los segundos usa usa un modelo babilolio de 60 porque la base era 60
Réplica
por ejemplo hay varias formas de contar puede ser de 5 de 10 de 20 usando las 4 cosas
sí porque en lo los árabes se inventaron los números actuales más o menos
y que eran 10 bueno 9 tiene cero sí pero pero la base romana era 5
no era 10 porque 10 eran 5 pero contaban de 5 en 5 la base romana eran 5 sí
pero pasa que los números romanos no valen para eso pedido ellos usaron 10 no sé por qué podían usar 20
o podían usar 60 o podían usar cualquier 02 los códigos informáticos son binarios el 1 2
podíamos usar varios varios códigos ellos que usaron 10 lo hicieron todo decimal no sé por qué podía ser además del 10
pero digo que intentaron racionalizarlo todo hasta las semanas de 10 días querían hacer
sí intento porque eso lo intentan hacer todos los Estados solo que ellos lo llevaron a cabo pero ustedes se fijan cómo definían el metro
es una barra de platino iridiado no no no les definían así el metro pues el metro es una barra de platino iridiado que hicieron que es la 10ª parte de un meridiano
una cosa así que hicieron con ellos una medida y esa medida la la guardaron en el museo de París y entonces es el estándar métrico decimal
querías sí sí la verdad sí yo tengo una pregunta relativa al al tema de la función social
Pregunta sobre "La función social de la Propiedad"
de la propiedad de la función social de la propiedad sí yo hace un par de un par de días en clase de derechos reales
pues el profesor comentó que la propiedad es un derecho real ilimitado eso significa que tú puedes
si tienes un piso pues lo puedes alquilar las veces que quieras pero el único límite
es la función social de la propiedad que dice la constitución en el en el artículo tal tal tal bueno la cuestión es que yo le pregunté
pero esto a qué se refiere de función social de la de la propiedad y dice bueno pues si tú tienes unos terrenos tienes
eh un las fincas uno con unos acres con unos árboles tú no tienes derecho a quemar
dicho bosque en tu propiedad porque bueno en cierto sentido estarías perjudicando la propiedad y tal
entonces mi pregunta es antes de la Revolución francesa cómo estaba regulado este derecho
porque a mí eso de que no tenemos un derecho real de propiedad sino uno de usufructo
yo diría que realmente tenemos derecho de propiedad pero la función social es para casos concretos
Réplica
la función de la propiedad valió hay un libro yo recomiendo de echar Pipes que se llama propiedad y libertad
"Propiedad y Libertad" (2002) - Richard Pipes
que explica un poco la evolución de la propiedad yo creo recordar que los derechos antiguos eran Yusu tendi Yus fruendy yusabutendi
derecho de uso derecho de disfrute del bien y derecho de abuso es decir que si yo tengo un bien
tengo derecho a usarlo al fruto de ese bien al fruto de mis de mis árboles o al fruto de de de lo que produzca ese terreno
y al derecho de abuso quiero decir que si el coche es mío lo puedo quemar si me da la gana
en en la propiedad central es el derecho obviamente es estúpido hacerlo igual que estúpido quemar un bosque
pero si el bosque es mío tengo derecho a quemarlo es lo que decía el derecho propiedad antiguo entonces le pones no aprenderse la propiedad social
se lo ponen por ejemplo con Air BNB aquí su alcaldesa decir que dices el piso no es suyo no lo puede alquilar
por qué no lo puede alquilar porque dice hay una función social de la propiedad entonces le van le van restringiendo el uso a su propiedad
o sea le pueden decir quién puede entrar en su casa o quién no puede entrar en mi país eh las casas no se pueden pintar de verde
porque dice que no es estético eso es real con la última normativa que se hicieron entonces en su casa sí es suya pero no le puede usar como quiere
puedo ponerle un piso más no no puedo poner un piso más si no molesta la desigual no le puede poner un piso más no sé qué de cosas de estética
del plan general de no sé qué cosas no puede usted poner un piso más su casa o no puede construir una casa al lado para que vean sus hijos
no puede por qué porque es la función social de la propiedad la función social de la propiedad como es un concepto um como le no sé cómo le llaman los abogados
ambiguo o inarticulado al final la interpreta el que gobierna en cada momento
o sea la función social de la propiedad la determina Rajoy la determina nado colado la termina Puigdemont la determina
quién está en cada momento en el mando y en qué consiste pues el gobierno anterior dejaba pintar las casas de verde ahora no deja
el gobierno ahora pintar de verde el conselleiro nuevo bueno pues y y porque antes se pudiera no
bueno porque ahora la función social ya cambió entonces no es un indeterminado no es un concepto claro
y al final quiere decir que la confianza es mía después va otra vez no au va
va el señor después va otra vez usted bedons eh buena tarde eh
Intervención
muy interesante a todos los males de los que ha faltado la Revolución francesa mis Yates podrían
la traición de la idea de igualdad que por ejemplo no se aplica a los dones la única Dred que le donen a Olin the Wish es
se ah ejecutada como un Oma pero no recunéis al dream de los dones ahora sí que es opta la defensa de la monarquía em
ya sé que ufa don's de una manera solapada y bien que es una es es un poder débil y es un poder don's que la gen no
no cabe de respecto a masa porque es difícil de poderlo justificar como el Woody apretés no de v i al Rey sensei nipensako miautiari
eh un religiosque que es iudio es que la eh mercadería mes eh sentí que mezcla la monarquía eh es la República
la forma que temes presente la mayoría de edad de loma Ayaka Khan eh Dew
eh don't shoes también fonomental la ilustración
Réplica sobre el sufragio universal y la expropiación del matrimonio
dos preguntas muy interesantes perdona que no no le parle catalán no le parle b eh pero muchas gracias por Parla
yo yo gasto mucho de escoltar gasto mucho qué bueno pues 10 ah pero no no no igual no lo entendía
uno de las dones del de las mujeres efectivamente el derecho el de eh
la constitución francesa la la segunda eh instaura el derecho de su faje universal nunca se aplica pero era masculino
y después otra cosa que que que apuntaban decir efectivamente y otra cosa que no conté decir el
el la Revolución francesa es propia el derecho al matrimonio es decir el matrimonio antes es una cuestión privada o religiosa
el el el estado de entonces monopoliza el monopoliza el
el matrimonio decide quién se puede casar cómo no se puede casar incluso otra vez en los derechos de contrato matrimonial por cierto no muy favorables a las mujeres
tampoco mejoró algo eso sí pero no no muy favorables a las mujeres tampoco pero fa es bueno
es parte de los problemas que hay ahora del del derecho del matrimonio homosexual etc. son debidos a que el matrimonio es estatal
si el matrimonio fuera privado cada persona se juntaría como quisiera desde el momento que el estado regula quién o no se puede casar pues ya el estado
ya se mete en la vida privada y sexual de la gente cosa que es una cosa de la muerte en francés en cuanto a la um
a la República la mayoría del del hombre eso ya no estoy tan de acuerdo la monarquía es proquistibilizatoria
La Monarquía es civilizatoria
porque la monarquía es una forma de gobierno que piensa muy largo plazo quiero decir la monarquía y la república se diferencian
una cosa el aparato del estado o el aparato de gobierno en un caso es de propiedad privada en la República
el aparato de gobierno es de propiedad pública eso quiere decir
si vive igual en una casa privada que una alquilada se cuida igual
entonces el aparato de gobierno va a estar igual de cuidado en una propiedad privada que una propiedad pública
vamos a ver el Rey tiene un aparte de pensar a largo plazo tiene interés directo en la prosperidad de su pueblo
porque el Rey como buen saqueador qué es el Rey va a saquear va a robar
igual que los gobernantes republicanos eh je je solo que el Rey va a robar y dice caray cuanto
si robo mucho mi pueblo se empobrece y si empobrezo me empobrezco yo y se empobrece
mi capital mi negocio entonces me interesa quitarme un puesto pero cuanto más rico sea mi pueblo más rico sea yo
la República piensa a corto plazo si yo enriquezco mi pueblo quién va se va a beneficiar va a ser el gobernante siguiente
la república se vende la República fue la las las la subo a la antigüedad también tenemos la República romana
es enormemente imperialista por cierto o no fue fuera fue César
toda aquella gente de antigua era gente enormemente imperialista y enormemente despiadada aparte que las elecciones estaban todas amañadas
sabemos todo así hubo repúblicas en la antigüedad de mucho ejemplo la República es una cosa que piensa a muy corto plazo
hombre claro las repúblicas que funcionaban bien las repúblicas serenísimas de Venecia por ejemplo ejemplos de la antigüedad pero no veo yo que a día de hoy por ejemplo
las monarquías europeas desluzcan con ninguna República Europea y donde se compara mejor
si es más más estable o no es en el mundo donde hay los dos modelos monarquías absolutas y repúblicas absolutas
que es el mundo árabe y cuáles son los países más estables del mundo árabe
el cuál es la la la la República el problema es su ventaja
que es muy popular por lo tanto la gente acepta el poder sin cuestionar
este es un poder enormemente legítimo porque nos dice que somos todos que participamos todos que todos el estado somos todos
que la hacienda somos todos y eso no es cierto siempre hay una clase de gobierno hay una clase dominada
yo cuando voy a Madrid por ejemplo que es la sede del poder estatal y paso por allí por el Palacio de la Moncloa
veo gente que va en coches muy buenos allí con cristales tintados va ha sido
los veo comiendo en restaurantes de lujo y cosas por el estilo y los veo que viven en Palacios de la Moncloa de la Zarzuela
de los Palacios Santa Cruz viven allí Palacios de verdad Palacios de verdad no casas grandes Palacios
y digo yo caray sí ve mira esta gente y tal y veo gente durmiendo en el suelo digo y vamos a ver pero no somos todos el estado
o hay unos que viven a costa de los otros
por eso digo la República es como nos quita las defensas frente al poder
yo no es que esté en contra República yo estoy en contra del poder me da igual que sea republicano o que sea monárquico digo que el el poder monárquico
tiende a ser más blando precisamente porque es más ilegítimo paradójico pero es así entonces
la gente se la enfrenta más la gente le hacía revueltas a los Reyes y no se les hace a las repúblicas es más difícil que se las saque
son como más legítimas y yo estoy en contra del poder no en contra me da igual una forma que otra y pienso que la forma
que menos ejerce del poder es la monarquía por eso es que la defiendo no es porque sea superior ni de origen divino ni nada así
la veo como más funcional y como más estable en el tiempo
perdone no no me ha contestado el tema de la mayoría de edad de de los humanos que sean fit grants y que cada escudo es libre
pero decidí para pensar eh los ciudadanos sí sí mayoría de edad que te podía
que antes no te podían reclutar y llevar a la guerra y después sí eso es la mayoría de edad del ciudadano o o cosas de la mayoría de edad
en dejarte participar en una especie de asamblea que solo podían votarlos solo podían votar los que tenían un censo
y después en la se había una segunda votación en la que ellos manipulaban quién podía salir y quién no qué mayoría de edad
es una mayoría de edad ficticia y repito a mí lo que me importa es la libertad la libertad que tenga yo real no la libertad que me digan los papeles que tengo yo
porque muchas veces confundimos y está todos los mitos de la revolución que lo comenté que es el mito del constitucionalismo
El mito del "Constitucionalismo"
parece que porque está escrito en un papel o un documento pues muy bien escrito que esté que tenemos derechos y
la declaración de los derechos del hombre que fue incorporado a la primera revolución a la primera constitución francesa que porque está escrito un papel
y los derechos nos vienen y antes no los teñimos vamos a ver los seres humanos antes tenían derechos igual y sabían cuándo había una injusticia
sabían cuándo había 1 1 mayoría de edad y y repito que todos estos escritores autores ilustrados
volter se criaron en un mundo supuestamente así que era muy de poca mayoría de edad y creo que escribieron
como algún problema que tuvo alguno con bastante libertad casi todos ellos voltear a una persona por ejemplo enormemente rica
ah entre otra cosa porque traficaba con negros pero una persona enormemente rica
eran personas que yo veía que podían publicar leer escribir ah o sea mornet mornet por ejemplo
"Los orígenes intelectuales de la Revolución Francesa" (1969) - Daniel Mornet
en un libro precioso que se llama la la los orígenes intelectuales de la Revolución francesa cuenta como el ambiente intelectual de la época
la las ciudades francesas antes de que antes de de haber constituciones tenían cenáculos tenían cafés tenían clubes
debatían del ateísmo los grandes libros activos franceses son de as de la Revolución francesa los dolvach por ejemplo
o del cura le o del del emetri por ejemplo del cura me salía son de antes de la canción francesa porque había cierta libertad expresión
eso qué me interesa a mí no que venga en un papel que yo soy un mayor de edad y después me trate como un imbécil
que creo que me lo hacen a mí muchas veces nuestros gobernantes queridos no prefiero que me digan mira no mando yo y punto
pero no me ande a decir que tengo derechos y después no los tengo porque lo otros es una especie de burla porque al final claro
si me tengo soy un mayor de edad eso es mucha cosa después como ganado como ganado me llevan a morir a guerras
contra gente que yo no conozco de nada no sé qué mayoría de edad es esa es una retórica es una discurso que cuaja muy bien porque repito
el estado moderno bebe de eso la Revolución francesa tuvo una fuerza enorme y luchar contra ellas eh
es muy difícil porque todos los imaginarios todas las ideas incluso ideas inteligentes que me hacen falta ustedes
están basadas en eso y que nos hicieron mayores de edad que nos dieron el libertario decían cuando éramos jóvenes ya
nuestro robinspier nos hizo libre hasta que estudias que aquí hizo robinspier no pero antes de robinspier éramos como una casta
que es clave vamos a ver cómo nos no cuadra no cuadra que haya todos esos escritos que haya todos esos libros
que haya todos esos tenáculos que haya todos esos clubes en un sitio tan atrasado de de de gente retrasada o que no era mayor de edad
no acabo de ver no acabo de ver yo eso entonces digo ese aspecto creo que más una retórica
fruta del discurso constitucional que de los hechos pero gracias por su pregunta
Pregunta
sí em bien usted ha planteado eh
ha contrapuesto la idea de centralización estatal con la idea de re eh regionalismo
fraccionamientos eh regional tanto en el marco de la Revolución francesa
como en el marco del absolutismo que hubo antes es decir ha puesto el ejemplo de los jacobinos en Francia
o el ejemplo de Felipe 5º en el caso de España imponiendo los de los decretos de de nueva planta
y lo que venía de antes la Revolución francesa lo lo exacerba exacto sí y lo que a mí me bueno
me confunde un poco la la cuestión de eh plantear ese regionalismo ese fraccionamiento territorial
como algo favorable a a la idea del liberalismo cuando liberalismo no de lo que soy yo
yo no soy liberal bueno ah je bueno pero por ejemplo del libre mercado tal vez sí
pero eh cuando precisamente precisamente cuando se centralizaba el poder muchas veces eh
se eliminaban aranceles internos fronteras internas que entorpecían el libre el libre comercio
es decir en el proceso de centralización absolutista hubo eh 1 1 proceso de liberalización también
porque se se eliminaron las fronteras internas sí pero se crearon fronteras externas
Las ventajas de la fragmentación política
se viralizó internamente pero se crearon a aranceles mayores cara afuera y y los grupos de y se centralizó la la política
ese nivel o sea los lobbies o los grupos de interés pues como que había de aquella época operaban a nivel estatal no a nivel local
saben cuál es el éxito de sitios divididos o casi independientes las las provincias unidas holandeses
qué eran provincias de hecho independientes con leyes distintas con normas distintas
absolutamente fragmentadas con fueros normas y fueron capaces de prosperar comercialmente al al al mejor nivel de Europa
mientras mantenían una guerra y derrotaban a la mayor potencia militar de Europa siendo 8 provincias que se llevaban mal entre ellas
y ni siquiera se hablaban precisamente porque estaban fragmentadas y se unían cuando les hacía falta
Pregunta
yo creo que se puede oponer a esa visión que tiene usted de la fragmentación como
un ejemplo favorable frente eh en relación al libre comercio por ejemplo analizando situaciones presentes en la actualidad
donde hay un mayor nivel de libre comercio dentro de los Estados Unidos un estado grande unificado o dentro de los pequeños países de Centroamérica
por ejemplo no solo entre ellos país entre ellos quiero decir
La desintegración política integra
entre ellos hay poco pero entre Estados Unidos dentro hay leyes de restricción de comercia interna también
pero menores yo es que hay entre los diferentes países de Centroamérica o de Suramérica sí
pero ahí entre ellos no hay muchos así les va pero quiero decir que no
el tamaño no influye o sea perdón el tamaño influye pero no es el único factor
yo a mí me gustaba la Europa fragmentada en 1 000 Estados una que era Europa que al final por estar fragmentada tenía un idioma oculto
una moneda y un derecho y una moral y estaba fragmentada en 1 000 unidades políticas
quería la Edad Media quería cuestionar precisamente esa apelación que hace la pregunta cuáles son los países más libres de Europa
económicamente cuáles son los grandes o los pequeños y los país de los países por ejemplo Alemania creo que hay 4 o 5
y Francia es creo que hay tres o 4 según cómo las miramos Alemania está Alemania está Austria está liachistein
y están están los cantones eh suizos y no sé si podemos contar a Holanda entre ellos
si se discute que cuál es el más cuál es el más pobre y el más grande cuál es el más rico el más es cierto que un país pequeño
tiene más incentivos a practicar el libre comercio sí porque no tiene ninguna capacidad de mantener 1 1 economía Antártica
como por ejemplo un hashtag país como Rusia en eso le doy la razón sin embargo tampoco es una constante
que los países pequeños practiquen el libre comercio Corea del Norte es un país extraordinariamente pequeño comparado con Estados Unidos
Albania es un país muy pequeño y sin embargo no ha sido históricamente la de antes la de antes
y sí sí claro también la de antes existió históricamente no ahora tampoco es que sea un paraíso libre de comercio
no es Singapur no no no Singapur que es un país pequeño también sí sí no
el tamaño no es el único factor pero si un país es pequeño y es cerrado
tiene que asumir los costes y los costes se ven claramente se ven los costes rápidamente que es una ruina el el país grande tipo Rusia
se ve a mal largo plazo etc. lo quiero decir es que la la desintegración política normalmente
integra no desintegra sí se integra realmente porque al final
en la Edad Media por ejemplo usted o yo podíamos estudiar en cualquier universidad de la cristiandad podíamos leer cualquier libro escrito
porque estaban todos en el mismo idioma precisamente porque estábamos fragmentados no había pasaportes el dinero tenía que ser único y de buena calidad
yo yo siempre por ejemplo hablo muy bien de las taifas que es una cosa que tiene muy mala presa en España
las taifas tenían el mejor dinero del mundo en su momento por ejemplo el mítico dinero del Rey lobo
porque no les quedaba otra eran tan pequeñas que pagaban con dinero de buena calidad o los demás no se lo querían y tenían eran libre
pasaron de ser sitios pequeños hace para la la taifa de Murcia pasarse un imperio comercial
durante los años que estuvo en taifa casi al nivel de de de Génova o de después decayó decayó por guerras o cosas así por el estilo
pero como como funcional funcionaba muy bien son las constelaciones de Estados que son las las repúblicas del de renacimiento
la las polis griegas que eran 100 polis y y nunca pudieron los persas con ellas
hola estas polis te digo que la fragmentación a mí me parece buena me parece muy buena y por en cuanto más pequeño sea
mejor tengo una enmienda que hacerle desde una perspectiva materialista histórica las taifas
prepoderaron sobre los reinos cristianos unificados o más unificados relativamente como Castilla
como la corona de Aragón o fueron arrasadas históricamente por esos pueblos más grandes yo creo que las
los árabes conquistaron España creo que en 10 años las taifas duraron un par de cientos de años
yo creo que si no fuera por las taifas serían derrotados en muy poco tiempo y esa otra cosa otra cosa tenían
tenían justicia privada dinero privado y ejércitos de mercenarios el famoso Cid Campeador era un
era un famoso mercenario de la taifa de Zaragoza pero fueron engullidas históricamente al final a largo plazo
te repito la España visigoda central tardó 10 años en conquistar por 35 000 árabes
cerca de una población de varios millones de disparos o guasmos cambia de aquella tampoco estaba unificada realmente España en aquel entonces
tenía un Rey solo y bastó con costarle la cabeza y acabar con él lo que quiero decir profesor basto
es que históricamente los pequeños ay no estaba integrado como ahora pero sí que tenía solo digamos 1 1
una cabeza digamos los pequeños pueblos fragmentados que usted defiende que tanto le gustan han sido engullidos por los grandes Estados
tiene que reconocer por lo menos la ventaja lo permites acabar verdad
sí si eres tan amable je je je no discúltalo tiene que reconocer la ventaja
por lo menos en el campo bélico de la centralización todos esos pequeños Estados alemanes
que a usted tanto le agradan cada vez que pasaba la gran armé tenía que tenían que rendirse inevitablemente no tenían capacidad de oponerse a Napoleón
Napoleón solo pudo ser derrotado por más enemigos de hecho Napoleón fue más eficiente en la gestión de sus recursos
materiales humanos que los enemigos pero como los enemigos lo superaban en número lo pudieron derrotar y no le dieron ningún premio de consolación
por haber sido más eficiente en el terreno bélico es más importante la eficacia que la eficiencia
Los pueblos sin Estado fueron más difíciles de conquistar
la pregunta es muy interesante porque es un tema que me me gusta mucho debatir aunque se escape se escapa un poco del tema
me gusta mucho debatir porque yo le pregunto por ejemplo 200 españoles conquistan el imperio Inca cara de varios millones
si en vez de haber un imperio Inca de 20 millones hay por ejemplo 10 Estados dos millones sería más fácil o más difícil
los en cambio comparemos con los indios sin estado las las confederaciones del norte los comanches por ejemplo
o incluso los mapuches o incluso dentro de América Central lacandones isqueles etc. otro tipo de pueblo
los pueblos sin Estados fueron infinitamente más difíciles de conquistar que los pueblos con estado descentralizado y con grande
o sea los grandes imperios el Azteca y el Inca cayeron en cuestión de semanas por un par de cientos de españoles el imperio visigodo y vosotros caen
caen de forma fácil por unos pocos cientos de españoles los Estados alemanes caen
qué pasó los alemanes caen a corto plazo
pero a España a la poleo que nos derrota son guerrillas los Estados centrales y luego de guerrillas que hay en Rusia
guerrillas que hay en otros sitios y otra cosa veríamos si si era capaz de manteneros sin a corto plazo sí después si es capaz de mantener el gobierno o no
y el caso de los presos del del de al de los Estados alemanes sí que los defiendo la Alemania de de chile entonces me parece mucho más interesante
que la Alemania de Hitler que es que es la Alemania unificada además las guerras que tenían entre ellos cuando eran guerras
pues eran guerras como mucho que causaban 10 20 muertos eran guerras casi a palos como las que hay en las fiestas ahora la no
pero en cuanto se juntan creo que la escala de la guerra cambió y no cambió para mejor
no necesariamente por eso les ponía el ejemplo de las polis griegas las polis griegas nunca los los persas que eran muy superiores
nunca pudieron con ellas fueron abrazadas por Filipo que era uno de ellos
después después después de que empezaran a corromperse a degenerar pero ellas el tiempo que duraron y plantaron cara
lo que fuera no hay no hay una prueba de eso no hay no hay prueba de conquistar un pueblo anárquico sea más fácil que conquistar a un pueblo arquico
hermano salud no no pero estoy encantado de hablar con él
je je je no hace muy buenos me interesa mucho esos temas aquí tengo tengo dos minutos eh
creo que tengo dos minutos sí sí y perdonen por por lápiz aquí le haré dos preguntas pero una muy breve
Pregunta
así que qué similitud desea usted entre am el socialismo científico y la Revolución francesa
por esto del culto a la razón y cuando hacían las casas de la ciencia y se reunían se reunían los domingos y era como crear una religión
exacto pero pero entre ellos y después no cree usted que una monarquía absolutista
hoy en día eh tenemos el ejemplo de Corea del Norte y no eso no es una monarquía
a ver esas ideas de la ciencia son más bien de los positivistas vienen después eh la Revolución francesa ejecuta a la washier
Réplica
el famoso químico y dice la revolución no precisa de sabios y le corta la cabeza supongo que lo que lo saben decir
eso es un culto que viene después pero es un culto a la razón pero efectivamente de ese culto de la ciencia el el marxismo sobre todo
Engel habla de un texto del del socialismo científico pero del socialismo utópico socialismo científico
y creo que le da prestigio con esa idea de la de la ciencia y la otra pregunta a la monarquía
coreana es que mona coreano es una monarquía porque es una república en sentido de que el reino o el Rey no es propiedad de esa familia
aunque intente serlo no tiene no se presenta a sí mismo como una dinastía propietaria de sus reinos
de hecho puede serlo pero no lo pero aún así ellos saben que el reino no es suyo no es como una dinastía de Reyes
que es que heredan por generación generación necesitan algún tipo de legitimación exterior pero intenta
intenta parecerse pero no lo es de hecho no se llama no se llama monarquía se llama República democrática de Corea
la democracia alemana se llama República democratic alemana no quita que ni sea república ni sea no
no no lo es de Alemania no lo es pero el el los principios de legitimación del régimen coreano son
pues representar la voluntad aunque sea de origen realmente de sangre lo lo que o sea una monarquía
lo que se está legitimada es por la sangre es el principio de primogenitura o el principio de sangre que fue en su momento un principio muy astuto
porque eliminaba las la guerra de sucesión se estaba claro en cada momento quién era el sucesor o sea pero los Reyes godos eran electos
eso quiere decir que cualquier Godo un poco listo decía este Rey es un burro voy a acabar con él me pongo yo en su sitio porque es
y hago que me elijan a mí de hecho ningún Rey Godo murió en la cama desde que se instauraron principios claros
de primogenitura diciendo eres tú y tú aunque mucho que me mates no vas a ser tú
aunque yo sea muy tal no voy a ser no me van a poder ser eso legítimo o eso la la República democrática de Corea
no está legitimada por sangre aunque sea por sangre no está legitimada por sangre sino por presentar la voluntad popular
o la voluntad del partido comunista una cosa así el fuchi sí
es parecido pero no lo es porque repito el principio de legitimación es distinto
una última pregunta y ya está venga es el tema de la charla pero esto es eh nada
Pregunta
pues como muchos premios y propuestas un capitalismo arácnido en este 20 por favor propuesta como subminiarismo
cualquier tipo de sistema capitalista es como que se puede mantener la materia privada
y me parece que hay experimentos con una policía o un o un sistema artístico la propiedad que son necesarias
Réplica
registro sí pero registro los puede haber los privados y los hay formas de registros privadas de la de la propiedad
simplemente listas y usted se anota allí y compiten y y simplemente usted se apunta allí
la propiedad realmente la defiende usted y la puede defender con policías públicas con policías privadas
pero yo cuando voy a un gran almacén o así veo que quien defiende la propiedad es una especie de señores que andan allí con uniformes y pueden defender la policía privada
yo voy a veces a urbanizaciones o a condominios veo que quien defiende aquellos son guardias de seguridad
más o menos privados en ese aspecto se puede defender otra cosa no es que se pueda
el debate que estamos planteando a ese respecto no sé si se puede o no si es legítimo o no y nosotros
no consideramos que esas policías sean legítimas por eso no nos gusta poder hacerlo pueden no es un problema de posibilidad y posibilidad
es un problema de legitimidad y legitimidad y eso es el debate que tenemos un debate entre anarquistas y minarquistas pero es un debate largo
Despedida
espero tenerlo alguna vez más con ustedes muchas gracias por

