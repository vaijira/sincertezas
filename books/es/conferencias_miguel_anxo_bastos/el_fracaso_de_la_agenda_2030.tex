\chapter{El fracaso de la agenda 2030.}

\chapterprecis{\href{https://www.youtube.com/watch?v=YnJU9JOAi0c}{Conferencia del profesor Miguel Anxo Bastos en el congreso anual del Xoán de Lugo de 2022.}}

\lettrine[lines=2, findent=3pt, nindent=0pt]{M}{uchas} gracias a todos. A ver, bienvenidos, un honor estar aquí con vos. A ver, primera conferencia igual no sale muy bien porque sólo tengo un mes para prepararla entonces a ver lo que sale de aquí.

Mi conferencia básicamente es sobre la transición energética. No voy a cuestionar si es necesaria o no. O sea voy a asumir los postulados digamos oficiales a todos los efectos sobre que es necesario una transición energética, los problemas del cambio climático. No lo voy a discutir básicamente porque no sé o sea no estoy en capacidad de discernir si una postura es la correcta o la contraria. Aquí hay gente que sabe más que yo del tema, después podrían comentar, pero no quiero entrar en ese punto. Yo quiero decir vale hay que hacer una transición energética lo que voy a discutir es si la forma que se está llevando la transición energética esta es a través de sistemas de planificación o a través de sistemas es la correcta o no para alcanzar ese fin, no estoy discutiendo el fin, el fin se puede discutir después en el debate o así, pero yo no quiero entrar en ese fin porque no estoy capacitado para decir si es necesaria no es necesaria si se va a calentar dos grados, seis u ocho o va a enfriar la temperatura porque no lo sé. Además yo sé, a ver no soy un completo ignorante en el tema, me gusta hablar del tema, pero veo que hay científicos muy cualificados en un lado, otros que lo cuestionan, voy a sumarme digamos a la postura oficial. No voy a no voy a cuestionar esto, repito, porque yo no soy capaz de discernir porque son temas muy complejos, son temas que hablan de albedos, hablan de grados y hablan de cosas de ese estilo ¿No? Y de eso no sé, entonces no sé cuál de los dos tiene la razón. Entonces voy a asumir que la postura digamos oficial es la correcta no tengo tampoco razones para dudarlo porque no sé. Entonces lo que voy a lo que voy a aplicar digamos y lo que sí entiendo más es de planificación, es un tema que me gusta mucho, de incentivos de políticas públicas y ese tipo de temas que sí que lo domino mejor ¿No? Pero antes de empezar asumiendo que repito voy a asumir a todos los efectos el discurso digamos de la ONU, el discurso de la agenda 2030 o el discurso de los de la del IPCC.

Lo voy a asumir en todo el caso, pero me gustaría hacer algunos caveats al respecto, es decir cuando se discuten estos temas a mí siempre me irritó mucho la idea de introducir conceptos de las ciencias naturales en las ciencias sociales. Siempre me irritó mucho, además lleva a muchas confusiones siempre, por ejemplo, son muy críticos con un concepto de igualdad usado a las ciencias sociales porque es un concepto yo creo que de origen matemático que tienen difícil la aplicación a la ciencia social o al concepto de fricción o al concepto de equilibrio o al incluso al concepto de evolución es un concepto que se usa para la evolución de los animales o de la naturaleza etc. Y creo que tiene, aunque se puede usar, creo que el encaje no es el mismo que en el ámbito natural. La evolución humana, la evolución cultural, tiene unos rasgos que son distintos de la otra.

También me fastidia bastante que se usen conceptos de las ciencias sociales, en este caso de la política, en el discurso científico. Por ejemplo, se habla consenso, el consenso es un concepto político consenso es cuando todos están de acuerdo y cuando digo todos es todos. Y como bien saben los polacos con la vieja dieta del siglo XVII con el liberum veto consenso era que cualquier diputado podía vetar, pero uno, bastaba uno. Y aquí la ministra el otro día le oí decir ``bueno es que 99\%...`` y el 99\% no es consenso es mayoría, todo lo reforzada que se quiere, pero es mayoría, hay un 1\% que discrepa.

Segundo se presume que la ciencia es democrática. Estamos acostumbrados a que hay muchas verdades o lo que es verdad y mentira se decide democráticamente, por ejemplo, una comisión de investigación en un parlamento decide por mayoría que tal señor es culpable o que la causa de tal catástrofe fue tal cosa, vale. Democracia está muy bien y tiene sus aplicaciones en unos ámbitos, pero no vale para decidir lo que es verdad y lo que es mentira, gana el mejor argumento. Pues en la ciencia pasa exactamente lo mismo que el 99\% diga una cosa y el 1\% diga otra cosa pues es un indicador, pero no es la prueba de que una cosa sea verdad. Simplemente es el 99\% contra el otro. Es más todos sabemos que en la historia muchas verdades asumidas por consenso fueron cuestionadas por un señor o una señora y ese señor o esa señora pues ganaron al final la batalla tras duro acoso porque la ciencia no es democrática es una cosa que tiene, no es el número es la calidad del argumento. Entonces es otra cosa que también se dice no es que la mayoría de los científicos solo el acoso, es más aunque fueran todos ellos eso no valdría tampoco. Aunque fueran todos los que digan una cosa digan lo mismo eso tampoco garantiza la verdad. Simplemente es que todos están de acuerdo en un momento dado, cualquier evidencia nueva, cualquier teoría nueva puede desbaratar todo el consenso que había anteriormente y la historia de la ciencia nos muestra todo eso. Yo digo, pero repito, muchas veces se usan discursos de corte político aplicados a la ciencia y eso para mí no es válido.

Por eso digo, se tiende a hipostatizar ahora que estamos en sede sagrada o se tiende a usar conceptos de la teología también a la política. La hipóstasis es un concepto que se usaba para los hombres el uno y el trino y esta cosa que se debatía mucho, pero ahora veo que se usa en muchos aspectos. Por ejemplo, el otro día leí un científico ofendido porque unas declaraciones de una política pues no sé qué decía del cambio climático que dice que se ofendía la ciencia. Vamos a ver, me ofenderá a mí que soy un científico, ofenderá a muchos científicos, pero no a la ciencia, señora que no tengo el gusto de conocer. Pero claro igual que el Estado igual que la clase obrera tenemos que decir o como decía el otro día que tal política del gobierno ofende a la nación española. Vamos a ver ofenderá a muchos españoles, pero no ofenderá a la nación española que tampoco tengo el gusto de conocer. Entonces que sé que es un concepto y no es una cosa que pretenda ofender, es decir, es un concepto que se usan en las ciencias sociales, pero que no vale. Pues aquí también la ciencia se usa de forma abstracta como si hablara una sola voz y fuera una señora que después de escuchar argumentos decir no. Claro la mayoría de los científicos opina así y esto es lo que vamos a hacer y estoy de acuerdo.

Igual que el concepto de experto. Los expertos, también decía otro señor otro día ``es que hay que hacer caso a los expertos en el cambio climático``. Vamos a ver, esto de experto es un concepto ambiguo, primero hay que definir experto en qué, es decir, cuáles en por ejemplo en estos debates climáticos o así, qué tipo de experto hace falta: el climatólogo o el biólogo que estudia, por ejemplo, como la oferta de los ecosistemas en el cambio climático, el economista tipo ejemplo Nicholas Stern, otros economistas que hablan de las consecuencias económicas del cambio climático incluso algunos politólogos como Klare que hablan de guerras de recursos por culpa del cambio climático otros hablan por ejemplo y entomólogos que hablan de que hay insectos o bichos que se puede aplicar a muchas cosas. Pero cuál es el experto relevante, qué expertos tenemos que hacerle caso en ese tipo de cosas y qué ponderación tienen que tener a la hora de elaborar un discurso político, eso es una cosa que no está clara. A mí me gusta escuchar a los expertos y sacar unas conclusiones, es decir, pero no hay un experto que domine sobre los demás expertos. Y segundo qué expertos son los correctos. Yo digo, por ejemplo, aquí en la pandemia sé que hay que escuchar a los expertos y sé que es una cosa correcta, pero la pandemia veo que el gobierno español como todos los gobiernos del mundo hacen una excepción escoge un grupo de expertos, ves los nombres y son expertos y yo no niego que sean expertos, no niego que sean de alta calidad, pero son expertos, por ejemplo, como los expertos fiscales que vi hace poco son expertos próximos al partido socialista. Lo cual no tiene nada de malo y no son menos expertos por eso, algunos de esos que le que leí allí son gente de mucha calidad, pero si hipotéticamente, no lo quiera Dios, gobernará yo alguna cosa o así y quisiera cazar expertos a lo mejor tomaba otro tipo de expertos y serían tan expertos como los de ellos, con tantas cualificaciones, títulos y otros como los otros. Entonces cuando escogemos el IPCC o una organización de ese estilo quién selecciona los expertos que van aquí a debatir, es una cosa que me gustaría decir, y con qué criterios se vota, si los escoge el ministro, la ministra... Repito no estoy negando la calidad de sus científicos, no estoy negando la calidad de sus informes sólo estoy diciendo que esos expertos no sé cómo fueron escogidos, cuál es el criterio que sea, quién decide entre los, yo qué sé climatólogos, cuál es el mejor de ellos, lo votan, se autoimpone, no sé cuál es el criterio para decidir. Supongo que será un criterio medio político medio de prestigio etc. Pero no es un criterio digamos tampoco objetivamente científico, sino que se introduce mecanismos de política en la ciencia también. Pero dicho esto voy a asumir el discurso, no tengo razón para dudar de él. Después de leer temas veo que buena parte de ellos que también dan argumentos muy potentes y no tengo razón para discutir.

Porque después otra cosa más, es decir, es que unos científicos están vendidos a la ciencia, están vendidos al capital, otros están vendidos a las petroleras, eso tampoco son argumentos. Yo puedo dar una conferencia a favor o en contra de algún tipo de temas y me paga la Shell, eso no altera la calidad en mi argumento, no la altera. Que yo lo haga por dinero o, al revés, puedo hacerlo por la más pura de las intenciones y estar equivocado y puedo hacerlo por puro dinero y acertar o decir lo correcto. Es que normalmente cuando llevas a una persona o así pues normalmente, claro, cuando tú escoges un experto escoges una persona así ya escoges alguien que ya previamente defiende tus puntos de vista, es decir, que esté una persona vendida o que esté así como a veces escucho también en debates de este tipo de cosas unos están vendidos o al otro, cualquiera de los dos puede estar vendido. Decir, pero no que están vendidos o que esté vendido y qué. Aquí me importa, una vez una vez que el argumento está expuesto, hay que analizar el argumento, no las razones por las que dio el argumento y a lo mejor queriendo estar vendido, queriendo ganar dinero, acierta, y a lo mejor queriendo ser puro falla. No ese es un argumento. Yo creo que en la ciencia o en el mundo de la ciencia lo que hay que discutir es lo que se dice o el argumento si es que sobre todo si está bien expuesto si está argumentado habrá que discutir el argumento ad rem no ad hominem. O sea hay que discutir sólo lo que se dice como decían los viejos filósofos, se discute ad rem, ad hominem y ad personam. Pues no, hay que discutir ad rem, sobre lo que estamos discutiendo, si lo dijo ese señor que está vendido al capital, el señor es una mala persona y tal... De acuerdo, pero estamos discutiendo de esto no estamos discutiendo sobre la persona. Dicho esto, repito, que como no tengo motivos, en principio, voy a asumir el discurso oficial y que hay que cambiar el clima y hay que reducir una serie de grados. Vale no lo voy a discutir, repito, tampoco sé.

El problema es cómo se hace para cambiar esos grados. Se optó por una solución estatista. Lo normal sería, la población, el consenso de la población mundial está de acuerdo en que es un problema gravísimo como lo leo yo allí, muchos coches y que va a haber una serie de consecuencias pues de ecosistemas, de hambrunas, de fenómenos climáticos extremos, de cosas así o sea que veo que hay un montón de consecuencias que van a afectar a buena parte de la humanidad. ¿Cuál sería lo lógico? Pues lo lógico es que gente que hace como en otras cosas pues cuando le dicen que hay un problema pues deja de consumir, la gente libremente dejaría de consumir fósiles, la gente libremente dejaría de consumir bienes, compraría coches eléctricos libremente, haría todo lo que sea necesario para todo esto. Eso es lo normal.

Aquí estaríamos en un debate que sería muy largo sobre, por ejemplo, ya filosófico, si hay un derecho de propiedad al clima, cuál es el clima correcto que queremos tener, si es el mismo para todos etc. Pero sería, uno, por un lado las personas que están descontentas con esta situación climática o de cambio lo normal es dejar de consumir y adoptar con consumos o formas de vida voluntariamente que lleven a eso, en otros ámbitos de la vida lo hacemos, aquí mi amigo me decía que dejara de fumar y cosas por el estilo, bueno pues es lo normal me avisa y tal y algo consigue y ya fumo menos que antes y cosas así, pero eso es lo normal. Y lentamente pues vamos acabando con el tabaco, vamos acabando con otro tipo de cosas, eso sería normal no y dejar de consumir y después en el ámbito legal, una vez decidido qué derechos tengo yo al clima pues establecer, esto ya es complicado, lo explica Walter Block y es, a día de hoy es muy difícil, pero bueno lo ideal sería pues establecer pleitos sobre las empresas que supuestamente alteran o dañan nuestro clima, empresas o particulares que, es decir, pues tal empresa echa mucho CO\textsubscript{2} o echa mucho dióxido de azufre o lo que sea, o cualquiera de esos gases que hay contaminantes, que tampoco sé muy bien cómo se llaman, pero bueno lo ideal sería pues que las personas dejáramos de usar esos bienes etc. y si pruebo que una empresa o algo así yo puedo hacer un pleito sindicado como se hizo con la banca con las cláusulas suelo que se juntaron muchos afectados presentaron un pleito y demandaron, no a la humanidad, demandaron a quién le hizo daño de verdad y entonces había un incentivo pues para que esa empresa que redujera estas cosas etc. Pero tienes que, en derecho, tienes que probar el daño y tienes que decir cuál es tu daño, qué daño te hizo y probar que fue esa empresa en concreto que te lo hizo. Lo normal sería eso, repito, adaptarnos voluntariamente a estas circunstancias, ya que estamos tan concienciados, y dejar de consumir y dejar de hacer estas emisiones, llevar una forma de vida más frugal, más simple, más verde. Eso sería normal.

Que me dirán, bueno, es que la gente no haría esto y tal, porque es muy egoísta y tal. O sea, pero primero está diciendo que la gente ese problema no lo ve tan problema. No que no diga que no es un problema, es como yo digo muchas veces mucha gente dice pues hay mucha hambre en el mundo, sí, pero sigues viviendo tan tranquilo. Y la gente te reconoce que hay un problema del clima, pero repito, yo por lo que veo no lo prioriza, porque repito, voluntariamente mucha gente no lo ve como un problema grave, que si fuera un problema grave lo vería, pero bueno, puede ser que no sea así. Entonces como no es así alguien más sabio que nosotros pues decide que hay que forzar. Digo porque además no se ven otros ámbitos, por ejemplo, yo digo vamos a ver si y aquí hay expertos inmobiliarios que me corrijan si estoy diciendo una barbaridad, ¿el precio de las casas en la playa está bajando de precio? ¿O más que los demás? ¿No? ¿Cómo puede ser eso si van a inundar unos años? ¿No? Tiene que ser así, yo si sé que mañana, en 10 años, va a haber un terremoto mi casa la vendo me voy a otro sitio, la liquido o no compro allí. Si yo sé que la costa se va a inundar los metros que dicen que se va a inundar Vigo, por ejemplo, que es mi ciudad, se va a inundar no sé cuántos leí hace poco, lo normal es que las casas a pie de playa o las casas cerca de la costa empiecen a descender de valor ya. Porque no hay que esperar a que se inunde, cualquier persona sabe que los precios de mercado descuentan, entonces digo algo falla y repito como cualquier persona que trabaja en sectores inmobiliarios debería saber no basta con que sean todos basta que sean unos pocos que empiecen a vender sus casas que confíen en todo esto, empezarían a vender sus propiedades y eso se notaría de alguna forma u otra en el precio. Y yo no lo veo. Antes de dar esta conferencia pregunté a unos amigos míos inversores o así que se dedican a hacer negocios de invertir cosas y así con labor los fondos verdes estos que hay voluntarios, que la gente puede comprar fondos para descarbonizar ¿Esos ganan en la bolsa? Estuve leyendo el libro de Peter Newell que es muy bueno el powershift, el mismo reconozco es un defensor de esto que los fondos verdes no dieron gran resultado y que la gente voluntariamente decide no invertir en ese tipo de cosas no sé si por bien o por mal si hacen bien o hacen mal la gente puede ser muy egoísta, pero digo que voluntariamente la gente acepta el problema, pero repito que no hace como dice Taleb en su genial libro ``jugársela bien`` no que a mí es lo único que me encanta decir no veo yo que ponga el dinero en actuaciones en lo que dice que hace, por lo tanto me da un poco que desconfiar. 

Pero bueno supongamos que aun así la gente es miope tiene miopía temporal y no sabe ver el problema o no lo acepte o cosas por el estilo. Tampoco lo hacen los gobiernos. Por cierto pues yo veo a Narendra Modi, el presidente de la de la India, dijo que iba a retrasar 20 años los objetivos. Xi Jinping de una forma críptica, como hacen los chinos al hablar, que no hay que desechar lo viejo antes de alcanzar lo nuevo, dijo una frase así que me gusta a mí mucho, quiero decir que no va a abandonar el carbón hasta que llegue el día ideal que todos ustedes carbonezcan. Además una de las sanciones, si no recuerdo mal, una de las sanciones que puso cuando fue lo de Taiwán fue romper los acuerdos de cambio climático. Vamos a ver los problemas climáticos a quién, por lo que me dicen a mí, a quién más van a afectar es a la India, China, países con monzones. ¿Cómo sus países son tan osados de maltratar a sus poblaciones? Ustedes no parece que tiemblen de miedo con ... 

Entonces digo vale, como la gente no quiere o no es capaz de llevar a cabo esto pues tienen que venir los gobiernos que sabe que tienen una información de corte superior al del común de los mortales a poner una serie de normas para hacer esto, repito que estoy asumiendo que hay que hacerlo, yo veo que por ejemplo ellos presumen un consenso que no veo yo a la hora de la verdad en otros aspectos, pero ojalá fuera así entonces como la gente no entiende, no ve el problema, no lo percibe, es necesario. Entonces, qué hace, pues hace una serie de medidas de política pública, que es donde más o menos estoy más cómodo yo, de intentar paliar ese cambio climático o esa agenda 2030 que no solo es climática, sino que tiene más aspectos y que digamos corregir las conductas a nivel global pues para solucionar el problema. Y el problema repito ante la solución ideal para cualquier problema de este estilo yo creo que son mejores las soluciones de mercado que las soluciones de estado. Aquí entramos en problemas de planificación central, no es una planificación central tipo soviética, es una planificación más bien indicativa tipo francés, con incentivos, con subvenciones, con regulaciones, con ciertas prohibiciones de conductas, pero digamos que no es una planificación central dura. Vera Lutz tiene un libro sobre la planificación francesa. Bueno, decía que fallaba, no fue tan clamoroso fracaso como la planificación soviética, pero fue un fracaso también, que normalmente las planificaciones no consiguen los objetivos por problemas supongo que todo buen austriaco conocerá que son problemas de información, problemas de cálculo económico. Y yo creo que en el problema de la planificación del cambio climático, la planificación de la respuesta para descarbonizar la sociedad creo que tiene errores de y digo como consejos si los gobiernos quieren hacer gastos no pueden planificarlo todo. Entonces, qué pasa, por ejemplo, los factores naturales no los pueden controlar. Lo mismo que, por ejemplo, la planificación central rusa cuando usted hace ustedes hacen un plan quinquenal vamos a ser somos el Gosplan vamos a hacer un plan quinquenal a 5 años vamos a producir tantos millones de patatas para tantos millones de patatas necesito tantos camiones para transportarlas tanto gasoil, tanta mano de obra, tantos hilos, tantas fábricas, las fábricas tienen que estar dimensionadas las fábricas de patatas fritas tienen que estar dimensionadas de acuerdo con la producción de patatas que tenía que hacer un plan ahí a 5 años, calcula producción estimada de patatas, cosas como estas, vale. Pero basta con que no llueva un año, no haya patatas te estropea el plan todo. Claro, tienes un montón de camiones allí parados, las fábricas están paradas, en la mano de obra sobra que por qué, porque hiciste un mal plan y sobre todo cambiar todo eso implica cambiar el plan por completo es uno de los problemas que tiene el plan, por ejemplo, factores de corte natural que tú no puedes contar con ellos. Pero claro te viene un volcán te viene una cosa así te altera todos los planes de cambio climático si lo cuantificas o sea hay que reducir tantas unidades de CO\textsubscript{2} que viene un volcán como el que vino el otro día o el de la Palma que es más pequeñito y te emiten unas cantidades desmesuradas de dióxido de carbono o de otros gases que te descolocan el plan, entonces el plan a lo mejor no parece la mejor solución.

Después, por ejemplo, cuando escoges las medidas tienes que seleccionar políticamente qué es verde y que no es verde. Primero que esas definiciones son arbitrarias porque en un principio pues será muy estricto, pues las renovables entendidas como hidroeléctrica, viento, sol, geotermia y corrientes marinas, bueno, todo lo que sea renovable digamos se acepta como verde, pero de repente te dicen que es verde la nuclear es verde, eso no está mal, de repente te dicen que el gas es verde o sea los bonos verdes pueden financiar industrias del gas o puede recibir subvenciones por gas. Entonces qué pasa qué clase de plan es este que está cambiando lo que es verde y lo que no es verde, no parece una cosa muy muy coherente. Pero esto desbarata los mercados y después claro pones toda una serie de incentivos con los impuestos. ¿Cuál es la cantidad de impuestos correcta?. Los impuestos verdes lo fija ese comité experto, fija un impuesto verde, pero cuál es el impuesto verde y el impuesto verde a qué tú empiezas al final a buscar excepciones, pues tal, si tú eres vulnerable tal etc. Entonces al final no pones impuestos a todo el mundo, sino que los pones a unos países y a otros no pues supuestamente por el desarrollo de esos países hay que exceptuarlos de esos impuestos o lo que sea así, pero claro estamos hablando de un problema global no de un problema de ayudar al desarrollo y decir que es un problema de segunda fila. Y después, por ejemplo, pues de estas cosas de los mercados de emisión de gases, pusiste una cantidad más una cantidad mudable tienes una cantidad que las empresas tienen que comprar de gas, de derechos de emisión, pues una cantidad supongo que la calcularán expertos, pero yo veo que la están cambiando también, los expertos fallan o la están usando políticamente que además están especulando con ellas.

Había un libro que me gusta mucho que es el de Boris Brutzkus ``Economic planning in Soviet Russia`` que fue el primer libro que apuntó el fenómeno de cálculo económico con el socialismo fue paralelo a Mises y yo creo que fue un poco anterior incluso, era un economista agrario ruso que vivió los tiempos del comunismo de guerra, los vivió en primera persona vivía en el país y aplicó técnicas de análisis aplicado próximas a la escuela austriaca, él era muy amigo de Hayek, usó elementos de análisis para explicarle lo que veía básicamente son falta de coordinación al no haber sistemas de precios había descoordinación total entre la producción de factores, por ejemplo, también descoordinaciones temporales y vio un montón de fallos en la producción. Un montón de fenómenos de descoordinación, por ejemplo, se proponen una serie de industrias, vale, pues esto hay que financiar una serie de industrias, pero no se tiene en cuenta si las energías que tienen que respaldar eso están suficientemente mal, es decir, ponemos unas nuevas medidas no vemos si el abandono de las viejas y el de las nuevas y la aparición de las nuevas están coordinados, no sé si me explico, por un lado quieres poner las nuevas, vale, está muy bien, pero por otro lado quieres abandonar las viejas, pero las viejas abandonan a un ritmo y las nuevas al otro y qué pasa que a veces pues hay un gap como dicen los economistas hay un lapso temporal. Y qué pasó ahora en esta crisis precisamente por una falta de coordinación se mete a derivar de no usar mecanismos de mercados qué pasó pues bueno que se empezaron a poner energías verdes se empezó a reducir, por ejemplo, el carbón en toda España, en Galicia pues se empezó a reducir otro tipo de energía, se redujo el coche, pero qué pasa las energías que pusimos nuevas renovables tienen un problema son intermitentes, es decir, que no puedes decidir primero usarlas todo el día porque a lo mejor ese mes no hay viento o lluvia que mueva las centrales hidroeléctricas o no hay suficiente sol o hay exceso de sol para usar la termosolar fotovoltaica, es decir, no hay suficiente energía para esto. Y tú no tienes el respaldo suficiente que es lo que está pasando porque abandonaste tus industrias antes, ahí sí Xi Jinping es un listo, no abandonar lo viejo hasta tener lo nuevo, pues nosotros abandonamos lo viejo, pero no pusimos lo nuevo. Entonces qué pasa, que hay un desfase y qué pasó pues que estos veranos y así no llovió o no hizo viento bastante, entonces qué pasa que hubo que echar mano del respaldo porque la energía al ser intermitente necesitas un sistema de respaldo que normalmente ese son centrales de ciclo combinado de gas. Claro como tiene que echar mano del gas, pero qué pasa, el gas es, por el sistema marginalista que tiene un nombre técnico que no me acuerdo ahora cómo es, básicamente el último factor de producción que entra es el que fija los precios cosa que es normal en economía siempre digo la gente se escandaliza con ese sistema, pero es un sistema bastante lógico económicamente, podíamos optar por otros, es un sistema bastante lógico económicamente por tanto no lo critico como tal por ejemplo ustedes ahora van por la calle van a Orense y van por el río mío y encuentran una pepita de oro de las que dejaron los romanos, bueno esa pepita de oro se la van a remunerar exactamente igual que la pepita sacada de una mina con mucho esfuerzo porque es la última unidad marginal la que fija el precio. Es que la pepita la sacaste sin esfuerzo y la otra la sacaste a costo cero y la pepita de la mina la sacaste con 40 mineros trabajando 40 horas, da igual. Igual que el petróleo que se saca del mar del Norte se paga exactamente igual que el de Nigeria independientemente de los costes que tenga o sea tiene una lógica. Pero cuando se adopta a este sistema porque antes este sistema no lo había aparte de que es lógico económicamente yo creo que también se usa para subvencionar a las renovables porque las renovables normalmente el coste variable, el coste de producción es cero, se producen a cero coste, una vez amortizadas o quitando los costes de instalación el coste variable de la energía al ser viento, al ser sol o así es gratuito normalmente el coste que se ofrecen es más barato que las otras, pero claro si se ofrece la de producción barata al mismo precio que la cara lo que estás haciendo es subvencionando la barata para eso se puso, como un mecanismo indirecto de subvención de las renovables, de forma que no se entere la gente, es una forma sutil, una forma de subvencionar que la gente invierta en renovables porque había cierta reticencia a invertir en ella entonces aparte de incentivos fiscales le pusieron incentivo en la retribución. Cuando las renovables no llegan salta el gas, estaba por las nubes por muchos motivos además se puso aún más por las nubes por la guerra y por otros factores dispara a los precios de electricidad entonces los gobiernos se asustan y dejan volver otra vez a las energías antiguas. Los austriacos dirían es reswitch de capital, es volver a un viejo debate de Böhm-Bawerk el reswitch, volver a formas de capital más primitivas cuando hay algún problema de ese estilo que se vuelve a veces a formas de capital más primitiva entonces se da un reswitch y que estamos quemando carbón no sólo en España, en toda Europa porque acá estamos volviendo a tecnologías más primitivas precisamente porque la nueva no fue suficiente.

¿Pero qué nos están diciendo con esto?. Aparte de que se ve que es un mecanismo claro de descoordinación temporal, se mete por no ajustarse el mercado unas empresas con otras haciendo el cambio como sí se habría hecho si fuera un mecanismo más de mercado. Es decir, mira, de antes de la crisis es más importante bajar el precio del gasoil o autorizar formas de producción de necesidad menos limpias, pero más baratas, primero para garantizar el suministro y segundo para pues para garantizar que no suba mucho el precio. Que estás diciendo entonces que el cambio climático no es tan importante, eres tú el que lo está diciendo. Yo escuchaba el otro día al secretario general de la ONU, señor Guterres, estamos acelerando hacia el infierno decía, hacia el infierno climático. Decía estamos acelerando hacia el infierno climático, pero vamos a ver si estamos acelerando y estamos quemando carbón o sea que por encima en el futuro habrá que limpiar más aún de lo que limpiamos, o sea que no acabo de entender, estás diciendo tú mismo que no es tan importante, pero no lo digo yo, repito que yo acepto el discurso, lo dicen ellos que no es tan importante, que puede esperar, que no es una cosa que estemos acelerando tan rápido hacia el infierno que aún puede esperar unos añitos, pues vamos a pasar unos añitos con esto y después cuando pase unos añitos limpiaremos todo. Cosa que no es tan fácil porque si tú haces una empresa de este estilo no te reabre una central solo por unos meses, querrá un contrato, querrá unas garantías a medio plazo, no es que te abro y te cierro cuando me da la gana.

Después la gente el ciudadano si escucha esto pierde el sentido de urgencia, pero no quedamos que lo más urgente que había era esto, que era una cuestión vital, existencial, de guerras, de hambres, de cosas por el estilo y por una subida en la electricidad no somos capaces los europeos de soportarla por el bien del planeta. Dicen que es una medida impopular, los gobiernos hacen estas cosas porque hay elecciones, pero vamos a ver pues digo que a mí estas cosas me sorprenden porque son muy contradictorias. Porque digo vamos a hacer esto, pero vamos a ver no quedamos en que la población mundial o la población española acepta todo. Además cuando alguien discrepa se le echan en los twitters y lo comen y cosas por el estilo. Vale, estoy de acuerdo que hacen bien criticar el twitter, pero entonces cómo puede ser una medida que reduzca el consumo de energía, que suba el precio de gasoil, que debe ser lo debe ser lo ideal porque así se consume menos y empezamos a asumir los costes del cambio climático como buenos europeos, con costes en la calidad de vida. Pero vamos a ver en qué quedamos si la población quiere esto querrá más de esto, lo popular será darle más, más subidas al gasoil o más subidas al carbón. Entonces cómo puede ser impopular, lo popular será lo contrario, que estamos cargándonos el planeta. Pero o falla algo aquí o no es tan popular, no sé digo yo, es lo que están diciendo ellos con sus actuaciones, no yo que a mí me parece bien estas cosas, entonces claro hay estos problemas, y el tipo de energía que se escuchó y el tipo de energía renovable que se usa que tiene problemas.

A ver, hay un libro que a mí me gusta mucho, mucho además, que es el de Andreas Malm es un ecomarxista se llama ``Capital fósil``, es uno de los libros más interesantes que vi, pero también es uno de los libros más autosubversivos que vi en mi vida, nunca vi algo tan autosubversivo, porque él explica, por ejemplo, en el siglo XIX había como dos opciones o el carbón o la energía hidráulica para mover molinos y cosas por el estilo, vale, y Malm explica por qué se optó por la energía de carbón, porque la energía hidráulica es muy buena, es muy barata y es una maravilla, qué pasa, que es intermitente, hay días que no hay agua entonces tienes que parar las máquinas, tienes que parar los batanes y sobre todo tienes que localizar la industria sólo donde había un río o cerca de un río, bueno tiene que estar en un río. La de carbón es continua es más contaminante obviamente y es más cara, pero es continua siempre te garantiza, es modulable si te puedes darle más gas o menos gas según compensa, cosa que la otra no al no depender de ti no puedes hacerlo y puedes colocarlo donde quieras. Y explicaba que todo eso, le echa la culpa a los salarios mínimos, fíjese, pues dice al poner los salarios tan altos pues las empresas que tenían hidráulica no podía permitirse el lujo de tener al trabajador parado una serie de horas mientras no había agua en el río entonces esas medidas de salarios al final castigaron energía hidráulica. Yo creo que algo semejante le pasan a esas energías renovables que tienen problemas de este estilo, pero bueno yo creo que es una maravilla es los grandes inventos de la humanidad sacar energía del viento, sacar energía del agua, sacar energía del Sol, es una gran maravilla, es un complemento, una cosa positiva, que puedas sustituirlo del todo eso lo veo yo más difícil por otros dos factores, que ya pasaron en el siglo XIX, por la modularidad, es decir, que puedas modular la potencia a gusto más o menos o eso creo que sólo puede hacer la hidráulica o las centrales de bombeo que parece que son muy populares porque hay que duplicar, hay que crear como dos embalses en muchas zonas, entonces los problemas que hay, hay mucha oposición.

Bueno la idea es esta, que la energía renovable pues puede ser un buen complemento, yo creo que el mercado la adoptaría perfectamente sin problema ninguno mas es una maravilla, qué pasa que la dimensión correcta la determina el mercado también con ensayos de prueba y error se va determinando qué cantidad de energía hidráulica, solar, geotérmica o eólica pues un sistema eléctrico, un mix eléctrico soportaría, complementaría y lo haría. Pero esta idea de querer cambiarlo todo sin pensar en si es posible yo creo que es uno de los fallos de la planificación, simplemente porque tú planificas una cosa, pues repito es algo que ya vio Brutzkus, que había falta de coordinación, había tractores pero le faltaban baterías, había zapatos pero le faltaban cordones. No claro, es que tú tienes que pensar en todo y lo que más gracia me hace, estaban los barcos de pesca rusos listos, con pescadores, faltaban anzuelos nadie pensó en ellos entonces los barcos no podían salir. Precisamente, simplemente basta con un señor en Moscú se olvide, para que todos los barcos estén sin anzuelo, basta con eso.

Entonces claro yo creo que el planificador puede planificar, Vlacav Smil que es un autor que me gusta mucho, aunque discrepo con él en algunas cosas, pero bueno, es un autor que me gusta mucho, cuando habla en ``Energy transitions`` de transiciones energética dice que el sistema energético que tenemos ahora es la mayor obra jamás creada por el hombre, todo el sistema energético, todo lo la infraestructura de hidrocarburos, toda la infraestructura de oleoductos, de refinería y otras cosas por el estilo es la mayor obra que jamás creó el ser humano, el querer cambiar todo eso así, en una serie de años es muy arrogante. Entonces tiene muchos problemas y tantos problemas repito que son derivados de la coordinación. Yo creo que en sistemas de mercado para conseguir la transición energética serían mejores, de hecho hay un paper en el ``The Quarterly Journal of Economics`` o no sé si leí alguno en el ``Journal of Libertarian Studies`` hace tiempo sobre este tema, pero sí, si quisieras hacerlo tendrías que usar este tipo de mecanismos. Pero claro, antes tienes que conseguir que la gente quiera hacerlo, si la gente quiere hacerlo mano de santo, ya está hecho, el problema es que falla algo o la gente no debe querer o algo así o no le debe dar tanta importancia como dice.

Pues yo como buen austriaco creo en la preferencia revelada, es una cosa que siempre me gustaba mucho de Mises, es decir no es lo que yo diga, Mises ponía un ejemplo, la gente dice que prefiere la Biblia al alcohol, pero Mises decía que no veía tiendas de biblias de 24 horas que veía otro tipo de cosas. Entonces la gente dice que prefiere la Biblia al alcohol, pero con realmente consume más alcohol que biblias, es lo que decía Mises. Entonces en economía lo que importa no es lo que dice la gente es lo que hace la gente, por desgracia. Y creo que aquí pasa algo de eso, pues creo que los esfuerzos deberían ser en convencer a la gente de necesidad de cambio, pero esos esfuerzos de convencer a la gente del cambio ellos mismos los desmienten, no con lo que dicen sino con lo que hacen. Entre otras cosas, por ejemplo, unas medidas que tomaron, bajaron los impuestos 20 céntimos, te lo venden como una ayuda del gobierno, realmente es una reducción fiscal. Pues bajaron los impuestos al gasoil, pues menudo lucha contra el cambio climático. Digo yo que a mí me parece pues que dentro de su propio discurso me parece incoherente decir está diciendo que es más importante que los transportistas estén contentos que salvar a Gaia. No lo digo yo repito que yo asumo, están diciendo ellos, o sea a mí me gusta que mis gobernantes como decía con el alcohol piensen en el bien común y piensen en el futuro de la humanidad.

Bien, vale, ahora voy a hablar un poco de una transición concreta que es la de la movilidad eléctrica, que también es otro fenómeno de este estilo. Aquí me inspiro mucho en teóricos de crecimiento para mí son de lo más interesante que hay.  El otro día hablamos en una comida de Turiel, ``el petrocalipsis``, después, por ejemplo, si hay algún físico así pues el libro de Alicia Valero de ``Thanatia`` que habla de lo que dicen los especialistas es que se van a acabar los recursos, se van a acabar todos los materiales críticos, buena parte de ellos, los necesarios para hacer pues para hacer tierras raras y cosas de esas, que se pueden estar acabando, que el capitalismo colapsará y sobre todo que es imposible ninguna transición energética. Turiel o Carlos Taibo, repito son de los libros más interesantes que leí en mucho tiempo de politólogos aunque no esté de acuerdo con ellos están bien escritos, se argumentan cosas por el estilo y yo saqué muchas ideas de ahí supongo que a ellos no sé si les gustará, pero que os comenté aquí. Pero me pareció sinceramente de lo más interesante que leí en mucho tiempo todo este tipo de libros de Kunstler o del decrecimiento, del colapso, me interesa mucho porque están diciendo una cosa que la transición energética no es posible por mucho que te pongas y están diciendo argumentos, que no hay materiales suficientes, estas cosas de que son intermitentes se lo leí a alguno de ellos, fue donde descubrí yo cuál era el problema de la renovable, que bueno supongo que ellos lo sabrán mejor que yo porque yo no soy científico, pero claro que tiene un problema de modularidad, tiene un problema de que se interrumpe, una industria no puede interrumpir su hospital, no se puede interrumpir la industria, en general, no se puede parar mucho tiempo sin gran problema. Entonces ellos demostraban y con muchos datos que va a haber un colapso precisamente porque no hay sustituto al fósil, no hay sustituto. Todo lo que están diciendo que es imposible de hacer no sé si es correcto o no, pero me pareció muy interesante.

Pero poco a poco la movilidad eléctrica pues es otro ejemplo en este caso de planificación, tenemos que transformar los coches de combustión en coches eléctricos,  es una idea que también puede parecerme bien es una idea pues razonable etc. pues para acabar con el cambio climático. Y aquí creo que la planificación es como más concreta es como más estricta y lo peor pusieron un plazo yo creo que es lo peor que se puede hacer en política pública poner un plazo, eso creo que muchas veces los politólogos o los analistas que toman estas decisiones a lo mejor prima el juicio del experto, el juicio del experto pues en clima o así enfrente a otro tipo de cosas. Entonces hay una serie de problemas primero de descoordinación, yo lo que estoy leyendo es, de nuestros autores que acabo de citar, ahora es un problema de descoordinación que tú quieres hacer una transición pues hacia el coche eléctrico, pero te falta toda la logística del coche eléctrico. El coche eléctrico puedes tenerlo, falta toda la cantidad, por ejemplo, de litio, de baterías o capacidad de producción de baterías que hay no sé si se podrá hacer igual sí, pero de momento no la hay, no hay, por ejemplo, puntos de recarga, no hay ni siquiera probablemente capacidad eléctrica, capacidad instalada puede ser que la haya, pero producción de electricidad en el momento puede ser que no haya para la subida de consumo que tendría el coche eléctrico. Primero, puede ser coche eléctrico, pero como pones el coche eléctrico no eres capaz de desarrollar al mismo tiempo toda la logística necesaria y toda la logística necesaria, no es solo que haya puntos de carga, sino que haya todos los repuestos, que haya las baterías, que haya talleres de reparación, que haya otro tipo de cosas. Esa cosa normalmente Smil dice que las transiciones energéticas de pasar del carbón al camión o del ferrocarril al camión pues que llevaron mucho tiempo porque no es sólo poner una fecha, tienes que montar gasolineras, tienes que hacer repuestos, tienes que formar no sólo trabajadores sino trabajadores mecánicos que sepan repararte todo eso, y tienes que montar, por ejemplo, una red de carreteras para que te puedan andar los coches bien y eso es lo que falta aquí. Puedes pones una fecha, pero va antes probablemente la imposición del coche, digo imposición porque se pone por motivos coercitivos, no es una cosa voluntaria, o coche eléctrico o coche de hidrógeno, que está aún menos desarrollado, es teóricamente posible, pero la logística del hidrógeno no es tan fácil tampoco, no está desarrollada aún. Entonces primero haces una cosa, pero no coordinas el resto de los de los factores, entonces qué pasa y a lo mejor ni siquiera tienes capacidad productiva aunque hay que cambiar grandes decenas de millones de coches entonces no es tan fácil, tú haces una cosa y ya dices se abaratarán las baterías, sí históricamente la mayor parte de los bienes con el tiempo se abarata, cuidado, siempre y cuando el número de gente que lo adoptes es suficiente, es decir, la mayor parte de los bienes pues si tiene por lógica del mercado tienden a mejorar, pero más aún si es toda la humanidad que lo adopta. Pero yo lo que estoy viendo es que se está adoptando en unos sitios y no en otros, es decir, otros países no están adoptando de hecho estos modelos de movilidad eléctrica por tanto las mejoras de escala, las mejoras de abaratamiento no se dan, el teléfono móvil, por ejemplo, se abarató muchísimo, el teléfono móvil se produce a nivel mundial, tenemos un Huawei, una cosa de este estilo, tenemos un teléfono que se produce a nivel mundial, con escala a nivel mundial, una tecnología a nivel mundial y que vende pues a todo el mundo o a buena parte del mundo. Si nosotros nos quedamos con un sistema tecnológico para nuestra movilidad que es exclusivo nuestro esas ventajas se pueden dar, pero no la misma escala de abaratamiento que se pueden dar en otros bienes. Les puede pasar como Brasil con el etanol pusieron el etanol e hicieron allí unos coches con etanol tienen un mercado grande de etanol, pero como el resto del mundo no adoptó el etanol una parte de las ventajas se perdieron y al final van lentamente abandonando el etanol, no es tan ventajoso como era porque el resto del mundo no los copió entonces todas las ventajas de producción, de división del trabajo mundial etc., no se dieron.

Entonces falta todo esto, falta un montón de no sólo el coche eléctrico, el coche eléctrico a lo mejor lo tenemos, las baterías veremos si las tenemos, pon que las tenemos, lo que dice Turiel y esto es que no hay litio bastante por lo menos a corto plazo para atender la demanda de todas estas baterías. Ojalá podamos disfrutar de un magnífico coche eléctrico. Hay cambios en la geografía industrial, por ejemplo, también eso no se tiene en cuenta. La gente dice no van a ser las mismas fábricas que hacen los coches ahora las que hagan en el coche eléctrico o no. No fueron las fábricas de locomotoras las que empezaron a fabricar camiones, fueron otras fábricas nuevas y estamos viendo que los punteros en este tipo de tecnologías no están aquí, están en Estados Unidos con todas las problemas que pueden tener los puntos de recarga rápida, los coches eléctricos vienen de Estados Unidos tipo Tesla o cosas de este estilo, es decir, no vienen de aquí. Quién nos garantiza que van a ser las industrias de aquí las que siguen haciendo los coches, quién nos garantiza que nuestros puertos, nuestras infraestructuras son las adecuadas para producir esos coches eléctricos, todas las fábricas de componentes, de baterías, todas esas cosas van a ser instaladas aquí o van a estar instaladas en otros sitios, no lo sé. Entonces probablemente haya cambios de geografía industrial, probablemente no sólo que destruyas una industria, destruyas ciudades enteras, pero pasó otras veces en el mundo, en el norte de Inglaterra que era industrial se desindustrializó por cambios en tecnología y ahora es el sur el rico, todas aquellas empresas de Detroit o cosas así pues desaparecieron y están en el sur, esto probablemente traiga cambios no sólo es que vamos a hacer lo mismo las cosas.

Después hay un problema muy grave que es de algunos de los gobernantes pues no son austríacos no sé si para bien o para mal, pero no tienen claro el concepto de estructura de capital, es decir, todos sabemos lo que es una estructura de capital que cuanto más larga es pues más productiva es. La estructura de capital, es decir, la producción de bienes o servicios pues requiere de un tiempo fabricar un palo lleva un día, fabricar un hacha de piedra lleva una semana, fabricar una sierra eléctrica lleva un año o cualquier producción cualquier cambio a estos niveles que operamos nosotros hablamos de años, de muchos años de cualquier proceso productivo que tenemos.  Qué pasa, por ejemplo, pues ahora en 2030 tal no se pueden hacer, vale, qué pasa, por ejemplo, con las escuelas de ingeniería ya, no es que el problema se verán entre 5 años, no señor, el capitalismo anticipa los problemas y tú le pusiste un plazo artificial a ojo y ese yo creo que es el problema le he puesto un plazo 2035, 2040, lo que sea, pero le pones un plazo a ojo desde entonces cómo te pones una especie de condena a muerte empieza a operar la empieza a operar la economía, por ejemplo, empieza a operar en los primeros factores, es decir, un ingeniero joven, un científico joven dice quién se va a poner a estudiar el motor de combustión. Estamos hablando de 20 años, usted ahora es un ingeniero de 20 años dice yo estoy investigando motores de combustión tengo muy avanzadas las cosas, abandonas el estudio, pero descapitalizas, por ejemplo, a todos los profesores de mecánica o de combustión de todo tipo de costos, los ingenieros que saben hacer cosas los descapitalizas ya, no dentro de 20 años, lo que usted vale de profesor no vale para nada es mejor que cierre la facultad o se dedique a estudiar otra cosa si se puede. A veces algunas personas pues tienen sus conocimientos en ese tipo de tecnologías y la nueva pues a lo mejor no están preparados para entenderla o empiezan como niños pequeños digamos a estudiar el nuevo mecanismo yo no sé de eso, pero en otros ámbitos puede ser. O sea, pero un en mi ámbito, por ejemplo, si ahora me dicen que el socialismo es falso todos los profesores socialistas quedan inmediatamente descapitalizados, pero inclusive en este caso no es una crítica o los keynesianos o a los austriacos ahora me demuestran con una cosa que es falso lo que estoy diciendo con lo que mi vida no vale para nada, es más y si me reciclo tendría que empezar como un niño pequeño, a ver el capital a ver así no, todo este tipo de cosas así, mercancía dinero mercancía, tendría que empezar de cero porque mi capital no valdría nada, pues lo mismo le pasa a los ingenieros, a los expertos o a los centros de investigación que tienen esto, esos estudios se abandonan, y cuidado, todas las potenciales ventajas que tendría ese sistema, es decir, cuando tú prohíbes taxativamente y compras otras cosas, todas las tecnologías que se están inventando en este momento para reducir emisiones, por ejemplo, estas cosas que llaman AdBlue, AdBlue es una cosa que reduce muchísimo las emisiones, pero por qué no toleran ese tipo de cosas. Si tú prohíbes el gasoil, por ejemplo, sé que técnicamente es posible, sé que no está desarrollado aún, por ejemplo, la mezcla de hidrógeno con diésel que se puede hacer en la industria del shipping hicieron un estudio el capítulo está desarrollado, se puede mezclar una proporción con un simple adaptador en los motores diésel, se pueden adaptar los motores diésel que existen y adaptarlos a andar con hidrógeno también, eso está eliminando. Todo el encendido por microondas, por ejemplo, que es una cosa que elimina muchísimas partículas, está eliminado todo porque esos motores ya no pueden existir. Entonces todas las innovaciones posibles que podrías hacer, todo ese capital que está invertido en eso desaparece, por qué, porque prohibiste ese motor. Entonces todos aquellos que están dedicándose a eso literalmente o se dedican a otra cosa o empiezan a investigar todo eso.

Pero no sólo eso, cuál es el problema del gasoil ahora, uno de ellos mejor dicho, pues que no hay capacidad de refino. Pero usted imagine que tiene una refinería le dicen que en 2035 no va a haber diesel, es que hasta 2035 queda mucho, claro sí espere usted. ¿Usted Compraría una máquina nueva para cuando se le estropeara la vieja?. ¿Compraría una máquina nueva para su refinería? ¿Invertiría en su refinería? ¿La modernizaría su eficiencia? ¿Intentaría que emitiera menos? ¿Intentaría mejorar la calidad del producto o refinar cosas nuevas?. Empezará lentamente a desinvertir. Por lo que leí el 10\% de las refinerías en Europa ya cerraron como es normal. Entonces la estructura de capital no es sólo la producción del bien es un proceso muy largo que empieza desde el ingeniero que diseña la máquina, hasta toda la logística de apoyo, todo lo que son combustibles, repuestos, reparación, es todo el proceso que es muy largo lo que decía Böhm-Bawerk que cuanto más larga es esa estructura más productivo se es. Nuestro problema es que nuestras estructuras de capital son muy largas, son muy productivas, pero son muy largas entonces los daños no se empiezan a ver dentro de 15 años ni siquiera dentro de 10 se empiezan a ver ya porque los precios de la tierra donde están las refinerías dicen eso no lo quiero yo, todos los que tienen los créditos, todos los que tienen avales con ese tipo de cosas empiezan a tener problemas. Es un problema que tú tienes un bien que valía para jugar usted garantía como un crédito o cosas por el estilo y ahora resulta que ese bien que tienes no te vale nada, no te vale nada por una decisión política, es decir, tu refinería ya no vale nada o vale para chatarra lo que tiene el material y está pasando, muchas refinerías en mayor o menor escala pero está pasando. Y unos problemas del diésel es que no hay capacidad de refino y parece que el diésel es el que tiene más problemas, el diésel viene de Rusia, se está quemando gasoil en centrales de térmicas, no aquí, pero en otros países sí se está quemando y fuel como el chapapote del Prestige se está quemando de lo más contaminante que hay para sacar energía, pero aquí no pero en otros países sí. Y esto qué es, quiero decir, estamos cambiando todo esto, la estructura de capital de ser tan larga no afecta dentro de 20 años afecta ya y a lo mejor de eso no se dieron cuenta. Pero afecta a sectores imprevisibles desde universidades que a lo mejor tienen que cambiar títulos o asignaturas o hasta a fábricas de todo tipo. Si usted tiene una gasolinera invierte en ella, renovarla, por ejemplo, a lo mejor invierte de muy poca calidad. Y la sentencia de muerte que es para esos trabajadores porque los trabajadores más calificados es para decir mira espera este sector yo tengo 25 años, cómo que voy a trabajar en una refinería ya no vas a trabajar allí, ya no pido una hipoteca, ya no compro una casa porque sé que me voy a tener que mudar. Eso es una condena casi a muerte de tu trabajo a lo mejor se podría hacer una transición al coche eléctrico si se hiciera sin estas prisas, pero si se hiciera de otra forma no si se hiciera a través de mecanismos de mercado y sobre todo cuál es el problema del coche eléctrico, por qué no triunfa. A ver cualquier bien o servicio triunfa todos pasamos de escribir cartas a mano a usar el correo electrónico, todos pasamos de usar el teléfono de teclas al teléfono móvil, por qué, porque el nuevo aparato el nuevo trasto nos traía algo nuevo, nos traía algún tipo de mejora. El problema, no sé cuál será el problema, supongo que será un problema de oferta, el problema es que el coche eléctrico no debe aportar al ciudadano pues la impresión de que mejora en algo, no lo sé, si no se te lo quitarían de las manos, es decir, los teléfonos móviles, los ordenadores en su momento y tal se los quitaban de las manos y no hizo falta ni que lo subvencionaran ni que pusieran un plazo ni nada por el estilo, en cuestión de años, por ejemplo, pasamos en cuestión de años se adoptó el teléfono móvil se adoptaron muchas otras tecnologías rápidamente y nos adaptamos a ellas. Por qué no pasa con esto,  qué le pasa a ese producto que hay, no sé por qué, el coche eléctrico por cierto es más antiguo que el de combustión, los primeros Ford eran eléctricos, pero algo fallaba por algo no triunfaría en esos momentos cuando esa tecnología es una tecnología relativamente evolucionada para otros motores, para lavadoras o así el motor eléctrico funciona bien porque no funciona en el auto qué es lo que falla, por qué no se percibe como una ventaja porque mientras no se percibe como una ventaja va a haber resistencia a usar por muchas medidas se den.

El problema es vamos a quitar ciudadanos, pero yo leí un ecologista y acabó que solo los achatarramientos de los coches, después hablando de obsolescencia planeada, el capitalismo tiene su obsolescencia planeada, me dicen. Quién mete la su licencia planeada a coches en perfecto uso, eso es un derroche y no tiene perdón de Dios, es un derroche quitarle al primal quitar a Pablo su coche y tirar coches en uso. Pues bueno es un derroche, vale. Pero por proponer un plazo lo normal sería que fueran cambiando lentamente, pero yo leí un ecologista y dice que sólo el hecho de achatarrar un coche consume más CO\textsubscript{2} que el otro. Y después hay otro fenómeno de coordinación que se acaba y que no se hizo, se está haciendo la transición al coche eléctrico sin haber cambiado el mix eléctrico. Quiero decir, hay países que sí el coche eléctrico es perfectamente neutro, yo leí un estudio, bueno hay estudios al respecto que se pueden discutir, en países como China que la producción de energía eléctrica es con carbón, un coche eléctrico consume más que un coche diésel porque no porque el coche eléctrico emita emisiones, pero sí la energía necesaria para cargar el coche eléctrico. Entonces no se cambió el mix al mismo tiempo que el coche eléctrico, están poniendo una cosa con un mix que es anticuado, entonces al final puedes tener un coche eléctrico. Incluso  podrías hipotéticamente en España, creo que no así, tener un coche todo verde que está contaminando más que un coche diesel de última generación al que ni siquiera le dejan probar, ni siquiera le dejan usarse. Por qué, porque se está haciendo mal, se está haciendo precisamente porque se quiere hacer de forma planificada y se quiere hacer por unos sabidillos que debe haber en Bruselas o así, que quieren planificar. Bueno pasa en los gobiernos de todos los Estados que lo aplauden y lo copian todo. Y no se discute repito yo no estoy criticando esto, digo que se puede hacer esto sin usar todo este tipo de parafernalias, usando otro tipo de métodos y ofreciendo, por ejemplo, un producto atractivo, haz que el coche eléctrico sea mejor que el coche de combustión y ya te lo quitan de las manos. Por qué, porque no son capaces de hacer esto, ahí estaría la cuestión no haría falta ni planificación ni 2030 ni 2040, ya todos tendríamos el coche, pero forzar todo esto a mí no me gusta, aparte que climáticamente lo que se discute es que achatarrar coches no es precisamente la medida más ecológica que existe porque además achatarrar uno en uso para crear uno nuevo y producir uno nuevo, un coche eléctrico produce más CO\textsubscript{2} que uno de combustión, el proceso de producción, no el total, cuida las baterías de litio, el níquel, todo lo que se necesita.

Bueno, esto lo quería explicar básicamente no estoy discutiendo el cambio climático, ojalá bajen las temperaturas, ojalá no se tengan esos problemas y estoy de acuerdo, pero digo cuál es la forma correcta de alcanzar ese fin. Y eso es lo que estoy diciendo que creo que estos mecanismos que usan los gobiernos no son los adecuados para alcanzar ese fin precisamente porque son sistemas de planificación, son sistemas de este estilo, habría que pensar idear mecanismos de mercado para esto. Venga está.
